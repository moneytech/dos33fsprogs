%\documentclass{article}

%\usepackage{graphicx}
%\usepackage{colortbl}
%\usepackage{multirow}

\newcommand*\rot{\rotatebox{90}}

\definecolor{color0}{rgb}{0.000,0.000,0.000}	% Black
\definecolor{color1}{rgb}{0.890,0.118,0.376}	% Magenta
\definecolor{color2}{rgb}{0.376,0.306,0.741}	% Dark Blue
\definecolor{color3}{rgb}{1.000,0.267,0.992}	% Purple
\definecolor{color4}{rgb}{0.000,0.638,0.376}	% Dark Green
\definecolor{color5}{rgb}{0.612,0.612,0.612}	% Grey 1
\definecolor{color6}{rgb}{0.078,0.812,0.992}	% Medium Blue
\definecolor{color7}{rgb}{0.816,0.765,1.000}	% Light Blue
\definecolor{color8}{rgb}{0.376,0.447,0.012}	% Brown
\definecolor{color9}{rgb}{1.000,0.416,0.235}	% Orange
\definecolor{color10}{rgb}{0.616,0.616,0.616}	% Grey 2
\definecolor{color11}{rgb}{1.000,0.627,0.816}	% Pink
\definecolor{color12}{rgb}{0.078,0.961,0.235}	% Bright Green
\definecolor{color13}{rgb}{0.816,0.867,0.553}	% Yellow
\definecolor{color14}{rgb}{0.447,1.000,0.816}	% Aqua
\definecolor{color15}{rgb}{1.000,1.000,1.000}	% White

%\begin{document}

\newpage

\section{The Lores Memory Map}

\subsection{Why is it so weird?}
%\begin{center}
%\begin{large}
%{\bf Notes on the Apple II Lores Memory Map}
%\end{large}

%{\bf Or: Why is the memory map so weird}
%\end{center}
The Apple II is very much
a TV-typewriter video-terminal that happens to have a 6502
processor attached to give the display something to do.
(This makes it similar to the SoC in a Raspberry Pi, which is
a large GPU with a small helper ARM processor tacked onto the side.)

The Apple II video display is so central, that it even affects the
CPU timings.
The CPU clock usually runs at 978ns, but every 65th cycle
it is extended to 1117ns to keep the video output in sync with the colorburst.
This is why the 6502 runs at the somewhat unusual average speed of 1.020484MHz.

Text mode and low-resolution graphics share the same 1k region of memory
from addresses {\tt \$400} to {\tt \$800} for Page1.
A straightforward setup would have a linear memory map where
location (0,0) would map to address {\tt \$400}, location (39,0) would map
to {\tt \$427}, and location (0,1) would be at {\tt \$428}.
That would make too much sense.

For low-res, the first complication is what is represented by each 
memory byte.
In text mode this is the ASCII value you wish to display, or-ed with
\$80 so the high bit is set.
Leaving the high bit clear does weird things like enable inverse 
(black-on-white) or flashing characters.
Setting address {\tt \$400} to {\tt \$C1}
would put an 'A' (ASCII {\tt \$41})
in the upper left corner of the screen.
In low-res graphics mode the two 4-bit nibbles are split and
interpreted as two blocks, one above each other.
In this case the the {\tt \$C1} would be a color 1 (red) block on top
and a color 12 (light green) block on the bottom.
The colors are NTSC artifact colors, formed by outputting the raw bit
pattern out to the screen with the color burst enabled.
You can try this out yourself from BASIC by running 
{\tt TEXT:HOME:POKE 1024,193} to see the text result, and
{\tt GR:POKE 1024,193} to see the graphics result.

That is not too bad so far.
The next complication is packing the 40-columns of characters into
video memory.
Sadly 40 is not a nice power of two, so any packing is going to 
be inefficient somehow with respect to addressing bits.
The compromise is to pack three 40-byte columns into 128 bytes,
wasting 8 bytes (the ``screen holes'').

This still might not be that weird, but then the address interleaving
comes into play.
Note that row~0 starts at {\tt \$480}, but row~1 starts at
{\tt \$480} (a diff of 128), not {\tt \$428} (a diff of 40)
as you might expect.
Address {\tt \$428} actually corresponds to row 16.

For example, see the sample image in Table~\ref{table:loresmap} and how
the address values are interleaved.
This same image is shown in Table~\ref{table:linear} as it would
appear if memory was read linearly.
To make things even more confusing, the image is scattered even
more completely across the physical RAM chips for reasons we will
describe below.

The reason for this craziness, as with most oddities on the Apple II,
turns out to be Steve Wozniak being especially clever.
Early home computers often used static RAM (SRAM).
SRAM is easy to use, you just hook up the CPU address and read/write lines
to the memory chips and read and write bytes as needed.

The Apple II instead uses dynamic RAM (DRAM), where each bit is stored in
a capacitor whose value will leak away to zero unless you
refresh it periodically.
Why would you use memory that did that?
Well SRAM uses 6 transistors to store a bit, DRAM uses only 1.
So in theory you can fit 6 times the RAM in the same space, leading
to much cheaper costs and much better density.

To avoid losing DRAM contents, you must regularly refresh the capacitors.
This involves reading each memory value out faster
than it leaks away.
DRAM reads are destructive,
so a read operation always reads out, recharges, then writes back
the original value.
Because of this you can avoid explicitly refreshing DRAM with a dedicated
circuit if you can guarantee you perform a read of each memory row
in the required timeframe.

Many systems could not do this, so there was separate
hardware to conduct the refresh.
Often this hardware would take over the memory bus and halt the CPU
while it was happening, slowing down the whole system.
This is true of the original IBM PC;
if you ever look at cycle-level optimization on the PC
you will notice the coders have to take into account pauses caused by
memory refresh (the refresh tended to be conservative so some coders
chose to live dangerously and make refresh happen less often to increase
performance).

% Wozniak's article in Byte magazine, May 1977 (Volume 2, Number 5) 
% Gayler: The Apple II Circuit description
%  15-bit video address, 6 horiz 9 vert, increments, repeating 60Hz
%  vert has 262 values, horiz has 65 (40 chars+25 horiz blank)
% value is loaded from proper place, and latched, 7 bits written out?
%  Crazy, normally the 6502 runs at 978ns, but every 65th cycle
%   it is extended to 1117ns to keep the video output in sync
%  Which is why the average CPU freq of apple II is 1.020484MHz
% 192 dots vertical.  70 blanking
% Understanding the Apple II by Sather
%   interleaving, but also to not leave excessive holes in map
% In interview in Sather book Woz says could have had contiguous
%   memory with 2 more chips.

Steve Wozniak realized that he could avoid stopping the CPU for refresh.
The 6502 clock has two phases:
during first phase processor is busy
with internal work and the memory bus is idle.
The CPU only accesses memory in the second phase.
The Apple II uses the idle phase to step through the video memory
range and updates the display.
To refresh the 16k (model 4116) DRAM chips you need to read each 128-wide
row at least once every 2ms.
By carefully selecting the way that the CPU address lines map to
the RAS/CAS lines into the DRAM you can have the video scanning
circuitry walk through each row of the DRAMs fast enough to
conduct the refresh for free. 
This works beautifully, but as a side effect you end up with the Apple II's
weird interleaved memory maps.

%
 %          654 3210
%0x400	00 000 1000 000 0000
%0x480	00 000 1001 000 0000
%0x500	00 000 1010 000 0000
%0x580	00 000 1011 000 0000
%0x600	00 000 1100 000 0000
%0x680	00 000 1101 000 0000
%0x700	00 000 1110 000 0000
%0x780	00 000 1111 000 0000
%0x428	00 000 1000 010 1000
%0x4a8	00 000 1001 010 1000
%0x528	00 000 1010 010 1000
%0x5a8	00 000 1011 010 1000
%0x628	00 000 1100 010 1000
%0x6a8	00 000 1101 010 1000
%0x728	00 000 1110 010 1000
%0x7a8	00 000 1111 010 1000
%0x450,0x4d0,0x550,0x5d0,0x650,0x6d0,0x750,0x7d0,
%
%127 values needed
%
%0000 0000 0000 0000 = $0000	
%...
%0011 1111 1000 0000 = $3f80

Wozniak said in a later interview that in retrospect he could have
gotten a linear video memory map at the expense of two more chips
on the circuit board.
Apparently when designing the Apple II he thought most people would use BASIC
which hid the memory map, and did not realize the interleaving would
be such a pain for assembly coders.

%So this is the reason for the ugly memory map.
This is why low-level text and lowres graphics routines
%and text code often
%It is also why Apple II graphics code must 
can be complex, using lookup tables and
read/shift/mask operations just to do simple plot operations.
Fully generic routines have to handle all the corner cases, which is why
the Mode7 demo cheats and the sprite drawing code only works
at even row offsets (as this makes the code smaller and simpler).

While this seems needlessly complicated, the hi-res graphics mode
is even worse that the mess described above.

\tabcolsep=0.11cm
\renewcommand{\arraystretch}{0.5}
\begin{tiny}
\begin{table*}
\caption{Apple II lores display, 40x48.  Note the interleaving of the row addresses.  Rows 40-47 are ASCII text being interpreted as graphic blocks.\label{table:loresmap}}
\centering
\scalebox{.8}{
\begin{tabular}{|l|l|c|c|c|c|c|c|c|c|c|c|c|c|c|c|c|c|c|c|c|c|c|c|c|c|c|c|c|c|c|c|c|c|c|c|c|c|c|c|c|c|}
\hline
& &\rot{\tt \$00} &\rot{\tt \$01} &\rot{\tt \$02} &\rot{\tt \$03} &\rot{\tt \$04} &\rot{\tt \$05} &\rot{\tt \$06} &\rot{\tt \$07} &\rot{\tt \$08} &\rot{\tt \$09} &\rot{\tt \$0A} &\rot{\tt \$0B} &\rot{\tt \$0C} &\rot{\tt \$0D} &\rot{\tt \$0E} &\rot{\tt \$0F} &\rot{\tt \$10} &\rot{\tt \$11} &\rot{\tt \$12} &\rot{\tt \$13} &\rot{\tt \$14} &\rot{\tt \$15} &\rot{\tt \$16} &\rot{\tt \$17} &\rot{\tt \$18} &\rot{\tt \$19} &\rot{\tt \$1A} &\rot{\tt \$1B} &\rot{\tt \$1C} &\rot{\tt \$1D} &\rot{\tt \$1E} &\rot{\tt \$1F} &\rot{\tt \$20} &\rot{\tt \$21} &\rot{\tt \$22} &\rot{\tt \$23} &\rot{\tt \$24} &\rot{\tt \$25} &\rot{\tt \$26} &\rot{\tt \$27} \\
\hline
& &\rot{0} &\rot{1} &\rot{2} &\rot{3} &\rot{4} &\rot{5} &\rot{6} &\rot{7} &\rot{8} &\rot{9} &\rot{10} &\rot{11} &\rot{12} &\rot{13} &\rot{14} &\rot{15} &\rot{16} &\rot{17} &\rot{18} &\rot{19} &\rot{20} &\rot{21} &\rot{22} &\rot{23} &\rot{24} &\rot{25} &\rot{26} &\rot{27} &\rot{28} &\rot{29} &\rot{30} &\rot{31} &\rot{32} &\rot{33} &\rot{34} &\rot{35} &\rot{36} &\rot{37} &\rot{38} &\rot{39} \\
\hline
\multirow{2}{*}{\tt \$400} &0 &\cellcolor{color0}&\cellcolor{color0}&\cellcolor{color0}&\cellcolor{color0}&\cellcolor{color0}&\cellcolor{color0}&\cellcolor{color0}&\cellcolor{color0}&\cellcolor{color0}&\cellcolor{color0}&\cellcolor{color0}&\cellcolor{color0}&\cellcolor{color0}&\cellcolor{color0}&\cellcolor{color0}&\cellcolor{color0}&\cellcolor{color0}&\cellcolor{color0}&\cellcolor{color0}&\cellcolor{color0}&\cellcolor{color0}&\cellcolor{color0}&\cellcolor{color0}&\cellcolor{color0}&\cellcolor{color0}&\cellcolor{color0}&\cellcolor{color0}&\cellcolor{color0}&\cellcolor{color0}&\cellcolor{color0}&\cellcolor{color0}&\cellcolor{color0}&\cellcolor{color0}&\cellcolor{color0}&\cellcolor{color0}&\cellcolor{color0}&\cellcolor{color0}&\cellcolor{color0}&\cellcolor{color0}&\cellcolor{color0}\\
\cline{2-42}
&1 &\cellcolor{color0}&\cellcolor{color0}&\cellcolor{color0}&\cellcolor{color0}&\cellcolor{color0}&\cellcolor{color0}&\cellcolor{color0}&\cellcolor{color0}&\cellcolor{color0}&\cellcolor{color0}&\cellcolor{color0}&\cellcolor{color0}&\cellcolor{color0}&\cellcolor{color0}&\cellcolor{color1}&\cellcolor{color1}&\cellcolor{color1}&\cellcolor{color4}&\cellcolor{color2}&\cellcolor{color2}&\cellcolor{color2}&\cellcolor{color4}&\cellcolor{color2}&\cellcolor{color2}&\cellcolor{color2}&\cellcolor{color0}&\cellcolor{color0}&\cellcolor{color0}&\cellcolor{color0}&\cellcolor{color0}&\cellcolor{color0}&\cellcolor{color0}&\cellcolor{color0}&\cellcolor{color0}&\cellcolor{color0}&\cellcolor{color0}&\cellcolor{color0}&\cellcolor{color0}&\cellcolor{color0}&\cellcolor{color0}\\
\hline
\multirow{2}{*}{\tt \$480} &2 &\cellcolor{color0}&\cellcolor{color0}&\cellcolor{color0}&\cellcolor{color0}&\cellcolor{color0}&\cellcolor{color0}&\cellcolor{color0}&\cellcolor{color0}&\cellcolor{color1}&\cellcolor{color1}&\cellcolor{color1}&\cellcolor{color1}&\cellcolor{color1}&\cellcolor{color0}&\cellcolor{color1}&\cellcolor{color1}&\cellcolor{color1}&\cellcolor{color4}&\cellcolor{color2}&\cellcolor{color2}&\cellcolor{color2}&\cellcolor{color4}&\cellcolor{color2}&\cellcolor{color2}&\cellcolor{color2}&\cellcolor{color0}&\cellcolor{color1}&\cellcolor{color1}&\cellcolor{color1}&\cellcolor{color1}&\cellcolor{color1}&\cellcolor{color0}&\cellcolor{color0}&\cellcolor{color0}&\cellcolor{color0}&\cellcolor{color0}&\cellcolor{color0}&\cellcolor{color0}&\cellcolor{color0}&\cellcolor{color0}\\
\cline{2-42}
&3 &\cellcolor{color0}&\cellcolor{color0}&\cellcolor{color0}&\cellcolor{color0}&\cellcolor{color0}&\cellcolor{color1}&\cellcolor{color1}&\cellcolor{color1}&\cellcolor{color1}&\cellcolor{color9}&\cellcolor{color9}&\cellcolor{color9}&\cellcolor{color9}&\cellcolor{color0}&\cellcolor{color1}&\cellcolor{color1}&\cellcolor{color1}&\cellcolor{color4}&\cellcolor{color2}&\cellcolor{color2}&\cellcolor{color2}&\cellcolor{color4}&\cellcolor{color2}&\cellcolor{color2}&\cellcolor{color2}&\cellcolor{color0}&\cellcolor{color9}&\cellcolor{color9}&\cellcolor{color9}&\cellcolor{color9}&\cellcolor{color1}&\cellcolor{color1}&\cellcolor{color1}&\cellcolor{color1}&\cellcolor{color0}&\cellcolor{color0}&\cellcolor{color0}&\cellcolor{color0}&\cellcolor{color0}&\cellcolor{color0}\\
\hline
\multirow{2}{*}{\tt \$500} &4 &\cellcolor{color0}&\cellcolor{color0}&\cellcolor{color0}&\cellcolor{color1}&\cellcolor{color1}&\cellcolor{color1}&\cellcolor{color9}&\cellcolor{color9}&\cellcolor{color9}&\cellcolor{color9}&\cellcolor{color13}&\cellcolor{color13}&\cellcolor{color13}&\cellcolor{color0}&\cellcolor{color0}&\cellcolor{color1}&\cellcolor{color4}&\cellcolor{color4}&\cellcolor{color4}&\cellcolor{color2}&\cellcolor{color4}&\cellcolor{color4}&\cellcolor{color4}&\cellcolor{color2}&\cellcolor{color0}&\cellcolor{color0}&\cellcolor{color13}&\cellcolor{color13}&\cellcolor{color13}&\cellcolor{color9}&\cellcolor{color9}&\cellcolor{color9}&\cellcolor{color9}&\cellcolor{color1}&\cellcolor{color1}&\cellcolor{color1}&\cellcolor{color1}&\cellcolor{color0}&\cellcolor{color0}&\cellcolor{color0}\\
\cline{2-42}
&5 &\cellcolor{color0}&\cellcolor{color0}&\cellcolor{color1}&\cellcolor{color1}&\cellcolor{color9}&\cellcolor{color9}&\cellcolor{color9}&\cellcolor{color13}&\cellcolor{color13}&\cellcolor{color13}&\cellcolor{color13}&\cellcolor{color12}&\cellcolor{color12}&\cellcolor{color12}&\cellcolor{color0}&\cellcolor{color1}&\cellcolor{color4}&\cellcolor{color4}&\cellcolor{color4}&\cellcolor{color2}&\cellcolor{color4}&\cellcolor{color4}&\cellcolor{color4}&\cellcolor{color2}&\cellcolor{color0}&\cellcolor{color12}&\cellcolor{color12}&\cellcolor{color12}&\cellcolor{color13}&\cellcolor{color13}&\cellcolor{color13}&\cellcolor{color13}&\cellcolor{color9}&\cellcolor{color9}&\cellcolor{color9}&\cellcolor{color9}&\cellcolor{color1}&\cellcolor{color1}&\cellcolor{color0}&\cellcolor{color0}\\
\hline
\multirow{2}{*}{\tt \$580} &6 &\cellcolor{color0}&\cellcolor{color0}&\cellcolor{color1}&\cellcolor{color9}&\cellcolor{color9}&\cellcolor{color13}&\cellcolor{color13}&\cellcolor{color13}&\cellcolor{color12}&\cellcolor{color12}&\cellcolor{color12}&\cellcolor{color6}&\cellcolor{color6}&\cellcolor{color6}&\cellcolor{color0}&\cellcolor{color1}&\cellcolor{color4}&\cellcolor{color4}&\cellcolor{color4}&\cellcolor{color2}&\cellcolor{color4}&\cellcolor{color4}&\cellcolor{color4}&\cellcolor{color2}&\cellcolor{color0}&\cellcolor{color6}&\cellcolor{color6}&\cellcolor{color6}&\cellcolor{color12}&\cellcolor{color12}&\cellcolor{color12}&\cellcolor{color13}&\cellcolor{color13}&\cellcolor{color13}&\cellcolor{color13}&\cellcolor{color9}&\cellcolor{color9}&\cellcolor{color1}&\cellcolor{color0}&\cellcolor{color0}\\
\cline{2-42}
&7 &\cellcolor{color0}&\cellcolor{color1}&\cellcolor{color9}&\cellcolor{color13}&\cellcolor{color13}&\cellcolor{color13}&\cellcolor{color12}&\cellcolor{color12}&\cellcolor{color6}&\cellcolor{color6}&\cellcolor{color6}&\cellcolor{color0}&\cellcolor{color0}&\cellcolor{color0}&\cellcolor{color0}&\cellcolor{color0}&\cellcolor{color0}&\cellcolor{color0}&\cellcolor{color0}&\cellcolor{color0}&\cellcolor{color0}&\cellcolor{color0}&\cellcolor{color0}&\cellcolor{color0}&\cellcolor{color0}&\cellcolor{color0}&\cellcolor{color0}&\cellcolor{color0}&\cellcolor{color6}&\cellcolor{color6}&\cellcolor{color6}&\cellcolor{color12}&\cellcolor{color12}&\cellcolor{color12}&\cellcolor{color13}&\cellcolor{color13}&\cellcolor{color13}&\cellcolor{color9}&\cellcolor{color1}&\cellcolor{color0}\\
\hline
\multirow{2}{*}{\tt \$600} &8 &\cellcolor{color0}&\cellcolor{color1}&\cellcolor{color0}&\cellcolor{color0}&\cellcolor{color0}&\cellcolor{color0}&\cellcolor{color0}&\cellcolor{color0}&\cellcolor{color0}&\cellcolor{color0}&\cellcolor{color0}&\cellcolor{color0}&\cellcolor{color0}&\cellcolor{color0}&\cellcolor{color0}&\cellcolor{color0}&\cellcolor{color0}&\cellcolor{color0}&\cellcolor{color0}&\cellcolor{color0}&\cellcolor{color0}&\cellcolor{color0}&\cellcolor{color0}&\cellcolor{color0}&\cellcolor{color0}&\cellcolor{color0}&\cellcolor{color0}&\cellcolor{color0}&\cellcolor{color0}&\cellcolor{color0}&\cellcolor{color0}&\cellcolor{color0}&\cellcolor{color0}&\cellcolor{color0}&\cellcolor{color0}&\cellcolor{color0}&\cellcolor{color0}&\cellcolor{color0}&\cellcolor{color1}&\cellcolor{color0}\\
\cline{2-42}
&9 &\cellcolor{color0}&\cellcolor{color1}&\cellcolor{color0}&\cellcolor{color0}&\cellcolor{color0}&\cellcolor{color0}&\cellcolor{color0}&\cellcolor{color0}&\cellcolor{color0}&\cellcolor{color0}&\cellcolor{color0}&\cellcolor{color0}&\cellcolor{color0}&\cellcolor{color0}&\cellcolor{color0}&\cellcolor{color0}&\cellcolor{color0}&\cellcolor{color0}&\cellcolor{color0}&\cellcolor{color0}&\cellcolor{color0}&\cellcolor{color0}&\cellcolor{color0}&\cellcolor{color0}&\cellcolor{color0}&\cellcolor{color0}&\cellcolor{color0}&\cellcolor{color0}&\cellcolor{color0}&\cellcolor{color0}&\cellcolor{color0}&\cellcolor{color0}&\cellcolor{color0}&\cellcolor{color0}&\cellcolor{color0}&\cellcolor{color0}&\cellcolor{color0}&\cellcolor{color0}&\cellcolor{color1}&\cellcolor{color0}\\
\hline
\multirow{2}{*}{\tt \$680} &10 &\cellcolor{color0}&\cellcolor{color1}&\cellcolor{color0}&\cellcolor{color0}&\cellcolor{color2}&\cellcolor{color2}&\cellcolor{color0}&\cellcolor{color2}&\cellcolor{color0}&\cellcolor{color2}&\cellcolor{color0}&\cellcolor{color2}&\cellcolor{color0}&\cellcolor{color2}&\cellcolor{color2}&\cellcolor{color0}&\cellcolor{color0}&\cellcolor{color3}&\cellcolor{color3}&\cellcolor{color3}&\cellcolor{color0}&\cellcolor{color3}&\cellcolor{color0}&\cellcolor{color3}&\cellcolor{color0}&\cellcolor{color3}&\cellcolor{color0}&\cellcolor{color0}&\cellcolor{color3}&\cellcolor{color0}&\cellcolor{color3}&\cellcolor{color3}&\cellcolor{color3}&\cellcolor{color0}&\cellcolor{color3}&\cellcolor{color3}&\cellcolor{color3}&\cellcolor{color0}&\cellcolor{color1}&\cellcolor{color0}\\
\cline{2-42}
&11 &\cellcolor{color0}&\cellcolor{color1}&\cellcolor{color0}&\cellcolor{color2}&\cellcolor{color2}&\cellcolor{color2}&\cellcolor{color0}&\cellcolor{color2}&\cellcolor{color0}&\cellcolor{color2}&\cellcolor{color0}&\cellcolor{color2}&\cellcolor{color0}&\cellcolor{color2}&\cellcolor{color2}&\cellcolor{color2}&\cellcolor{color0}&\cellcolor{color3}&\cellcolor{color3}&\cellcolor{color3}&\cellcolor{color0}&\cellcolor{color3}&\cellcolor{color0}&\cellcolor{color3}&\cellcolor{color0}&\cellcolor{color3}&\cellcolor{color0}&\cellcolor{color0}&\cellcolor{color3}&\cellcolor{color0}&\cellcolor{color3}&\cellcolor{color3}&\cellcolor{color3}&\cellcolor{color0}&\cellcolor{color3}&\cellcolor{color3}&\cellcolor{color3}&\cellcolor{color0}&\cellcolor{color1}&\cellcolor{color0}\\
\hline
\multirow{2}{*}{\tt \$700} &12 &\cellcolor{color0}&\cellcolor{color1}&\cellcolor{color0}&\cellcolor{color2}&\cellcolor{color0}&\cellcolor{color0}&\cellcolor{color0}&\cellcolor{color2}&\cellcolor{color0}&\cellcolor{color2}&\cellcolor{color0}&\cellcolor{color2}&\cellcolor{color0}&\cellcolor{color2}&\cellcolor{color0}&\cellcolor{color2}&\cellcolor{color0}&\cellcolor{color0}&\cellcolor{color3}&\cellcolor{color0}&\cellcolor{color0}&\cellcolor{color3}&\cellcolor{color0}&\cellcolor{color3}&\cellcolor{color0}&\cellcolor{color3}&\cellcolor{color3}&\cellcolor{color0}&\cellcolor{color3}&\cellcolor{color0}&\cellcolor{color3}&\cellcolor{color0}&\cellcolor{color0}&\cellcolor{color0}&\cellcolor{color3}&\cellcolor{color0}&\cellcolor{color0}&\cellcolor{color0}&\cellcolor{color1}&\cellcolor{color0}\\
\cline{2-42}
&13 &\cellcolor{color0}&\cellcolor{color1}&\cellcolor{color0}&\cellcolor{color2}&\cellcolor{color0}&\cellcolor{color0}&\cellcolor{color0}&\cellcolor{color2}&\cellcolor{color2}&\cellcolor{color2}&\cellcolor{color0}&\cellcolor{color2}&\cellcolor{color0}&\cellcolor{color2}&\cellcolor{color0}&\cellcolor{color2}&\cellcolor{color0}&\cellcolor{color0}&\cellcolor{color3}&\cellcolor{color0}&\cellcolor{color0}&\cellcolor{color3}&\cellcolor{color0}&\cellcolor{color3}&\cellcolor{color0}&\cellcolor{color3}&\cellcolor{color3}&\cellcolor{color3}&\cellcolor{color3}&\cellcolor{color0}&\cellcolor{color3}&\cellcolor{color3}&\cellcolor{color0}&\cellcolor{color0}&\cellcolor{color3}&\cellcolor{color3}&\cellcolor{color3}&\cellcolor{color0}&\cellcolor{color1}&\cellcolor{color0}\\
\hline
\multirow{2}{*}{\tt \$780} &14 &\cellcolor{color0}&\cellcolor{color1}&\cellcolor{color0}&\cellcolor{color2}&\cellcolor{color0}&\cellcolor{color0}&\cellcolor{color0}&\cellcolor{color2}&\cellcolor{color2}&\cellcolor{color2}&\cellcolor{color0}&\cellcolor{color2}&\cellcolor{color0}&\cellcolor{color2}&\cellcolor{color2}&\cellcolor{color2}&\cellcolor{color0}&\cellcolor{color0}&\cellcolor{color3}&\cellcolor{color0}&\cellcolor{color0}&\cellcolor{color3}&\cellcolor{color0}&\cellcolor{color3}&\cellcolor{color0}&\cellcolor{color3}&\cellcolor{color0}&\cellcolor{color3}&\cellcolor{color3}&\cellcolor{color0}&\cellcolor{color3}&\cellcolor{color3}&\cellcolor{color0}&\cellcolor{color0}&\cellcolor{color3}&\cellcolor{color3}&\cellcolor{color3}&\cellcolor{color0}&\cellcolor{color1}&\cellcolor{color0}\\
\cline{2-42}
&15 &\cellcolor{color0}&\cellcolor{color1}&\cellcolor{color0}&\cellcolor{color2}&\cellcolor{color0}&\cellcolor{color0}&\cellcolor{color0}&\cellcolor{color2}&\cellcolor{color0}&\cellcolor{color2}&\cellcolor{color0}&\cellcolor{color2}&\cellcolor{color0}&\cellcolor{color2}&\cellcolor{color2}&\cellcolor{color0}&\cellcolor{color0}&\cellcolor{color0}&\cellcolor{color3}&\cellcolor{color0}&\cellcolor{color0}&\cellcolor{color3}&\cellcolor{color0}&\cellcolor{color3}&\cellcolor{color0}&\cellcolor{color3}&\cellcolor{color0}&\cellcolor{color3}&\cellcolor{color3}&\cellcolor{color0}&\cellcolor{color3}&\cellcolor{color0}&\cellcolor{color0}&\cellcolor{color0}&\cellcolor{color0}&\cellcolor{color0}&\cellcolor{color3}&\cellcolor{color0}&\cellcolor{color1}&\cellcolor{color0}\\
\hline
\multirow{2}{*}{\tt \$428} &16 &\cellcolor{color0}&\cellcolor{color1}&\cellcolor{color0}&\cellcolor{color2}&\cellcolor{color2}&\cellcolor{color2}&\cellcolor{color0}&\cellcolor{color2}&\cellcolor{color0}&\cellcolor{color2}&\cellcolor{color0}&\cellcolor{color2}&\cellcolor{color0}&\cellcolor{color2}&\cellcolor{color0}&\cellcolor{color0}&\cellcolor{color0}&\cellcolor{color0}&\cellcolor{color3}&\cellcolor{color0}&\cellcolor{color0}&\cellcolor{color3}&\cellcolor{color3}&\cellcolor{color3}&\cellcolor{color0}&\cellcolor{color3}&\cellcolor{color0}&\cellcolor{color0}&\cellcolor{color3}&\cellcolor{color0}&\cellcolor{color3}&\cellcolor{color3}&\cellcolor{color3}&\cellcolor{color0}&\cellcolor{color3}&\cellcolor{color3}&\cellcolor{color3}&\cellcolor{color0}&\cellcolor{color1}&\cellcolor{color0}\\
\cline{2-42}
&17 &\cellcolor{color0}&\cellcolor{color1}&\cellcolor{color0}&\cellcolor{color0}&\cellcolor{color2}&\cellcolor{color2}&\cellcolor{color0}&\cellcolor{color2}&\cellcolor{color0}&\cellcolor{color2}&\cellcolor{color0}&\cellcolor{color2}&\cellcolor{color0}&\cellcolor{color2}&\cellcolor{color0}&\cellcolor{color0}&\cellcolor{color0}&\cellcolor{color0}&\cellcolor{color3}&\cellcolor{color0}&\cellcolor{color0}&\cellcolor{color3}&\cellcolor{color3}&\cellcolor{color3}&\cellcolor{color0}&\cellcolor{color3}&\cellcolor{color0}&\cellcolor{color0}&\cellcolor{color3}&\cellcolor{color0}&\cellcolor{color3}&\cellcolor{color3}&\cellcolor{color3}&\cellcolor{color0}&\cellcolor{color3}&\cellcolor{color3}&\cellcolor{color3}&\cellcolor{color0}&\cellcolor{color1}&\cellcolor{color0}\\
\hline
\multirow{2}{*}{\tt \$4A8} &18 &\cellcolor{color0}&\cellcolor{color1}&\cellcolor{color0}&\cellcolor{color0}&\cellcolor{color0}&\cellcolor{color0}&\cellcolor{color0}&\cellcolor{color0}&\cellcolor{color0}&\cellcolor{color0}&\cellcolor{color0}&\cellcolor{color0}&\cellcolor{color0}&\cellcolor{color0}&\cellcolor{color0}&\cellcolor{color0}&\cellcolor{color0}&\cellcolor{color0}&\cellcolor{color0}&\cellcolor{color0}&\cellcolor{color0}&\cellcolor{color0}&\cellcolor{color0}&\cellcolor{color0}&\cellcolor{color0}&\cellcolor{color0}&\cellcolor{color0}&\cellcolor{color0}&\cellcolor{color0}&\cellcolor{color0}&\cellcolor{color0}&\cellcolor{color0}&\cellcolor{color0}&\cellcolor{color0}&\cellcolor{color0}&\cellcolor{color0}&\cellcolor{color0}&\cellcolor{color0}&\cellcolor{color1}&\cellcolor{color0}\\
\cline{2-42}
&19 &\cellcolor{color1}&\cellcolor{color1}&\cellcolor{color0}&\cellcolor{color0}&\cellcolor{color0}&\cellcolor{color0}&\cellcolor{color0}&\cellcolor{color0}&\cellcolor{color0}&\cellcolor{color0}&\cellcolor{color0}&\cellcolor{color0}&\cellcolor{color0}&\cellcolor{color0}&\cellcolor{color0}&\cellcolor{color0}&\cellcolor{color0}&\cellcolor{color0}&\cellcolor{color0}&\cellcolor{color0}&\cellcolor{color0}&\cellcolor{color0}&\cellcolor{color0}&\cellcolor{color0}&\cellcolor{color0}&\cellcolor{color0}&\cellcolor{color0}&\cellcolor{color0}&\cellcolor{color0}&\cellcolor{color0}&\cellcolor{color0}&\cellcolor{color0}&\cellcolor{color0}&\cellcolor{color0}&\cellcolor{color0}&\cellcolor{color0}&\cellcolor{color0}&\cellcolor{color0}&\cellcolor{color1}&\cellcolor{color1}\\
\hline
\multirow{2}{*}{\tt \$528} &20 &\cellcolor{color1}&\cellcolor{color0}&\cellcolor{color0}&\cellcolor{color0}&\cellcolor{color0}&\cellcolor{color0}&\cellcolor{color0}&\cellcolor{color0}&\cellcolor{color0}&\cellcolor{color0}&\cellcolor{color0}&\cellcolor{color0}&\cellcolor{color0}&\cellcolor{color0}&\cellcolor{color0}&\cellcolor{color0}&\cellcolor{color0}&\cellcolor{color0}&\cellcolor{color0}&\cellcolor{color0}&\cellcolor{color0}&\cellcolor{color0}&\cellcolor{color0}&\cellcolor{color0}&\cellcolor{color0}&\cellcolor{color0}&\cellcolor{color0}&\cellcolor{color0}&\cellcolor{color0}&\cellcolor{color0}&\cellcolor{color0}&\cellcolor{color0}&\cellcolor{color0}&\cellcolor{color0}&\cellcolor{color0}&\cellcolor{color0}&\cellcolor{color0}&\cellcolor{color0}&\cellcolor{color0}&\cellcolor{color1}\\
\cline{2-42}
&21 &\cellcolor{color1}&\cellcolor{color0}&\cellcolor{color0}&\cellcolor{color0}&\cellcolor{color0}&\cellcolor{color0}&\cellcolor{color0}&\cellcolor{color0}&\cellcolor{color0}&\cellcolor{color4}&\cellcolor{color4}&\cellcolor{color4}&\cellcolor{color4}&\cellcolor{color4}&\cellcolor{color4}&\cellcolor{color4}&\cellcolor{color4}&\cellcolor{color4}&\cellcolor{color4}&\cellcolor{color4}&\cellcolor{color4}&\cellcolor{color4}&\cellcolor{color4}&\cellcolor{color4}&\cellcolor{color4}&\cellcolor{color4}&\cellcolor{color4}&\cellcolor{color4}&\cellcolor{color4}&\cellcolor{color4}&\cellcolor{color0}&\cellcolor{color0}&\cellcolor{color0}&\cellcolor{color0}&\cellcolor{color0}&\cellcolor{color0}&\cellcolor{color0}&\cellcolor{color0}&\cellcolor{color0}&\cellcolor{color1}\\
\hline
\multirow{2}{*}{\tt \$5A8} &22 &\cellcolor{color1}&\cellcolor{color0}&\cellcolor{color0}&\cellcolor{color0}&\cellcolor{color0}&\cellcolor{color0}&\cellcolor{color0}&\cellcolor{color0}&\cellcolor{color0}&\cellcolor{color4}&\cellcolor{color5}&\cellcolor{color5}&\cellcolor{color5}&\cellcolor{color4}&\cellcolor{color5}&\cellcolor{color5}&\cellcolor{color5}&\cellcolor{color5}&\cellcolor{color5}&\cellcolor{color4}&\cellcolor{color5}&\cellcolor{color5}&\cellcolor{color5}&\cellcolor{color4}&\cellcolor{color4}&\cellcolor{color4}&\cellcolor{color14}&\cellcolor{color4}&\cellcolor{color14}&\cellcolor{color4}&\cellcolor{color0}&\cellcolor{color0}&\cellcolor{color0}&\cellcolor{color0}&\cellcolor{color0}&\cellcolor{color0}&\cellcolor{color0}&\cellcolor{color0}&\cellcolor{color0}&\cellcolor{color1}\\
\cline{2-42}
&23 &\cellcolor{color1}&\cellcolor{color0}&\cellcolor{color0}&\cellcolor{color0}&\cellcolor{color0}&\cellcolor{color12}&\cellcolor{color0}&\cellcolor{color0}&\cellcolor{color0}&\cellcolor{color4}&\cellcolor{color5}&\cellcolor{color5}&\cellcolor{color5}&\cellcolor{color4}&\cellcolor{color5}&\cellcolor{color5}&\cellcolor{color5}&\cellcolor{color5}&\cellcolor{color5}&\cellcolor{color4}&\cellcolor{color5}&\cellcolor{color4}&\cellcolor{color5}&\cellcolor{color4}&\cellcolor{color9}&\cellcolor{color4}&\cellcolor{color4}&\cellcolor{color4}&\cellcolor{color4}&\cellcolor{color4}&\cellcolor{color0}&\cellcolor{color0}&\cellcolor{color0}&\cellcolor{color0}&\cellcolor{color0}&\cellcolor{color0}&\cellcolor{color12}&\cellcolor{color0}&\cellcolor{color0}&\cellcolor{color1}\\
\hline
\multirow{2}{*}{\tt \$628} &24 &\cellcolor{color1}&\cellcolor{color0}&\cellcolor{color0}&\cellcolor{color0}&\cellcolor{color12}&\cellcolor{color0}&\cellcolor{color0}&\cellcolor{color0}&\cellcolor{color0}&\cellcolor{color4}&\cellcolor{color5}&\cellcolor{color5}&\cellcolor{color5}&\cellcolor{color4}&\cellcolor{color5}&\cellcolor{color5}&\cellcolor{color5}&\cellcolor{color5}&\cellcolor{color5}&\cellcolor{color4}&\cellcolor{color5}&\cellcolor{color4}&\cellcolor{color5}&\cellcolor{color4}&\cellcolor{color4}&\cellcolor{color4}&\cellcolor{color5}&\cellcolor{color4}&\cellcolor{color5}&\cellcolor{color4}&\cellcolor{color0}&\cellcolor{color0}&\cellcolor{color0}&\cellcolor{color0}&\cellcolor{color0}&\cellcolor{color12}&\cellcolor{color0}&\cellcolor{color0}&\cellcolor{color0}&\cellcolor{color1}\\
\cline{2-42}
&25 &\cellcolor{color1}&\cellcolor{color0}&\cellcolor{color0}&\cellcolor{color12}&\cellcolor{color12}&\cellcolor{color12}&\cellcolor{color12}&\cellcolor{color0}&\cellcolor{color0}&\cellcolor{color4}&\cellcolor{color5}&\cellcolor{color5}&\cellcolor{color5}&\cellcolor{color4}&\cellcolor{color5}&\cellcolor{color5}&\cellcolor{color5}&\cellcolor{color5}&\cellcolor{color5}&\cellcolor{color4}&\cellcolor{color5}&\cellcolor{color5}&\cellcolor{color5}&\cellcolor{color4}&\cellcolor{color9}&\cellcolor{color4}&\cellcolor{color5}&\cellcolor{color4}&\cellcolor{color5}&\cellcolor{color4}&\cellcolor{color0}&\cellcolor{color0}&\cellcolor{color0}&\cellcolor{color0}&\cellcolor{color12}&\cellcolor{color12}&\cellcolor{color12}&\cellcolor{color12}&\cellcolor{color0}&\cellcolor{color1}\\
\hline
\multirow{2}{*}{\tt \$6A8} &26 &\cellcolor{color1}&\cellcolor{color0}&\cellcolor{color13}&\cellcolor{color13}&\cellcolor{color13}&\cellcolor{color13}&\cellcolor{color0}&\cellcolor{color0}&\cellcolor{color0}&\cellcolor{color4}&\cellcolor{color4}&\cellcolor{color4}&\cellcolor{color4}&\cellcolor{color4}&\cellcolor{color4}&\cellcolor{color4}&\cellcolor{color4}&\cellcolor{color4}&\cellcolor{color4}&\cellcolor{color4}&\cellcolor{color4}&\cellcolor{color4}&\cellcolor{color4}&\cellcolor{color4}&\cellcolor{color4}&\cellcolor{color4}&\cellcolor{color4}&\cellcolor{color4}&\cellcolor{color4}&\cellcolor{color4}&\cellcolor{color4}&\cellcolor{color0}&\cellcolor{color0}&\cellcolor{color13}&\cellcolor{color13}&\cellcolor{color13}&\cellcolor{color13}&\cellcolor{color0}&\cellcolor{color0}&\cellcolor{color1}\\
\cline{2-42}
&27 &\cellcolor{color1}&\cellcolor{color0}&\cellcolor{color9}&\cellcolor{color9}&\cellcolor{color9}&\cellcolor{color9}&\cellcolor{color0}&\cellcolor{color0}&\cellcolor{color0}&\cellcolor{color4}&\cellcolor{color4}&\cellcolor{color4}&\cellcolor{color4}&\cellcolor{color4}&\cellcolor{color4}&\cellcolor{color4}&\cellcolor{color4}&\cellcolor{color4}&\cellcolor{color4}&\cellcolor{color4}&\cellcolor{color4}&\cellcolor{color4}&\cellcolor{color4}&\cellcolor{color4}&\cellcolor{color4}&\cellcolor{color4}&\cellcolor{color4}&\cellcolor{color6}&\cellcolor{color4}&\cellcolor{color6}&\cellcolor{color4}&\cellcolor{color0}&\cellcolor{color0}&\cellcolor{color9}&\cellcolor{color9}&\cellcolor{color9}&\cellcolor{color9}&\cellcolor{color0}&\cellcolor{color0}&\cellcolor{color1}\\
\hline
\multirow{2}{*}{\tt \$728} &28 &\cellcolor{color1}&\cellcolor{color0}&\cellcolor{color1}&\cellcolor{color1}&\cellcolor{color1}&\cellcolor{color1}&\cellcolor{color1}&\cellcolor{color0}&\cellcolor{color0}&\cellcolor{color4}&\cellcolor{color5}&\cellcolor{color5}&\cellcolor{color5}&\cellcolor{color4}&\cellcolor{color5}&\cellcolor{color5}&\cellcolor{color5}&\cellcolor{color5}&\cellcolor{color5}&\cellcolor{color4}&\cellcolor{color5}&\cellcolor{color5}&\cellcolor{color5}&\cellcolor{color4}&\cellcolor{color4}&\cellcolor{color4}&\cellcolor{color4}&\cellcolor{color6}&\cellcolor{color4}&\cellcolor{color6}&\cellcolor{color4}&\cellcolor{color0}&\cellcolor{color0}&\cellcolor{color1}&\cellcolor{color1}&\cellcolor{color1}&\cellcolor{color1}&\cellcolor{color1}&\cellcolor{color0}&\cellcolor{color1}\\
\cline{2-42}
&29 &\cellcolor{color1}&\cellcolor{color0}&\cellcolor{color2}&\cellcolor{color2}&\cellcolor{color2}&\cellcolor{color2}&\cellcolor{color2}&\cellcolor{color0}&\cellcolor{color0}&\cellcolor{color4}&\cellcolor{color5}&\cellcolor{color5}&\cellcolor{color5}&\cellcolor{color4}&\cellcolor{color5}&\cellcolor{color5}&\cellcolor{color5}&\cellcolor{color5}&\cellcolor{color5}&\cellcolor{color4}&\cellcolor{color5}&\cellcolor{color4}&\cellcolor{color5}&\cellcolor{color4}&\cellcolor{color5}&\cellcolor{color5}&\cellcolor{color4}&\cellcolor{color6}&\cellcolor{color4}&\cellcolor{color6}&\cellcolor{color4}&\cellcolor{color0}&\cellcolor{color0}&\cellcolor{color2}&\cellcolor{color2}&\cellcolor{color2}&\cellcolor{color2}&\cellcolor{color2}&\cellcolor{color0}&\cellcolor{color1}\\
\hline
\multirow{2}{*}{\tt \$7A8} &30 &\cellcolor{color1}&\cellcolor{color0}&\cellcolor{color0}&\cellcolor{color6}&\cellcolor{color6}&\cellcolor{color6}&\cellcolor{color0}&\cellcolor{color0}&\cellcolor{color0}&\cellcolor{color4}&\cellcolor{color5}&\cellcolor{color5}&\cellcolor{color5}&\cellcolor{color4}&\cellcolor{color5}&\cellcolor{color5}&\cellcolor{color5}&\cellcolor{color5}&\cellcolor{color5}&\cellcolor{color4}&\cellcolor{color5}&\cellcolor{color4}&\cellcolor{color5}&\cellcolor{color4}&\cellcolor{color5}&\cellcolor{color5}&\cellcolor{color4}&\cellcolor{color6}&\cellcolor{color4}&\cellcolor{color6}&\cellcolor{color4}&\cellcolor{color0}&\cellcolor{color0}&\cellcolor{color0}&\cellcolor{color6}&\cellcolor{color6}&\cellcolor{color6}&\cellcolor{color0}&\cellcolor{color0}&\cellcolor{color1}\\
\cline{2-42}
&31 &\cellcolor{color1}&\cellcolor{color0}&\cellcolor{color0}&\cellcolor{color0}&\cellcolor{color0}&\cellcolor{color0}&\cellcolor{color0}&\cellcolor{color0}&\cellcolor{color0}&\cellcolor{color4}&\cellcolor{color5}&\cellcolor{color5}&\cellcolor{color5}&\cellcolor{color4}&\cellcolor{color5}&\cellcolor{color5}&\cellcolor{color5}&\cellcolor{color5}&\cellcolor{color5}&\cellcolor{color4}&\cellcolor{color5}&\cellcolor{color5}&\cellcolor{color5}&\cellcolor{color4}&\cellcolor{color4}&\cellcolor{color4}&\cellcolor{color4}&\cellcolor{color4}&\cellcolor{color4}&\cellcolor{color4}&\cellcolor{color4}&\cellcolor{color0}&\cellcolor{color0}&\cellcolor{color0}&\cellcolor{color0}&\cellcolor{color0}&\cellcolor{color0}&\cellcolor{color0}&\cellcolor{color0}&\cellcolor{color1}\\
\hline
\multirow{2}{*}{\tt \$450} &32 &\cellcolor{color1}&\cellcolor{color0}&\cellcolor{color0}&\cellcolor{color0}&\cellcolor{color0}&\cellcolor{color0}&\cellcolor{color0}&\cellcolor{color0}&\cellcolor{color0}&\cellcolor{color4}&\cellcolor{color4}&\cellcolor{color4}&\cellcolor{color4}&\cellcolor{color4}&\cellcolor{color4}&\cellcolor{color4}&\cellcolor{color4}&\cellcolor{color4}&\cellcolor{color4}&\cellcolor{color4}&\cellcolor{color4}&\cellcolor{color4}&\cellcolor{color4}&\cellcolor{color4}&\cellcolor{color4}&\cellcolor{color4}&\cellcolor{color4}&\cellcolor{color4}&\cellcolor{color4}&\cellcolor{color4}&\cellcolor{color7}&\cellcolor{color0}&\cellcolor{color0}&\cellcolor{color0}&\cellcolor{color0}&\cellcolor{color0}&\cellcolor{color0}&\cellcolor{color0}&\cellcolor{color0}&\cellcolor{color1}\\
\cline{2-42}
&33 &\cellcolor{color1}&\cellcolor{color0}&\cellcolor{color0}&\cellcolor{color0}&\cellcolor{color0}&\cellcolor{color0}&\cellcolor{color0}&\cellcolor{color0}&\cellcolor{color0}&\cellcolor{color4}&\cellcolor{color4}&\cellcolor{color4}&\cellcolor{color4}&\cellcolor{color4}&\cellcolor{color4}&\cellcolor{color4}&\cellcolor{color4}&\cellcolor{color4}&\cellcolor{color4}&\cellcolor{color4}&\cellcolor{color4}&\cellcolor{color4}&\cellcolor{color4}&\cellcolor{color4}&\cellcolor{color4}&\cellcolor{color4}&\cellcolor{color4}&\cellcolor{color4}&\cellcolor{color4}&\cellcolor{color4}&\cellcolor{color5}&\cellcolor{color0}&\cellcolor{color0}&\cellcolor{color0}&\cellcolor{color0}&\cellcolor{color0}&\cellcolor{color0}&\cellcolor{color0}&\cellcolor{color0}&\cellcolor{color1}\\
\hline
\multirow{2}{*}{\tt \$4D0} &34 &\cellcolor{color1}&\cellcolor{color0}&\cellcolor{color0}&\cellcolor{color0}&\cellcolor{color0}&\cellcolor{color0}&\cellcolor{color0}&\cellcolor{color0}&\cellcolor{color0}&\cellcolor{color5}&\cellcolor{color4}&\cellcolor{color4}&\cellcolor{color4}&\cellcolor{color4}&\cellcolor{color4}&\cellcolor{color4}&\cellcolor{color4}&\cellcolor{color4}&\cellcolor{color4}&\cellcolor{color4}&\cellcolor{color4}&\cellcolor{color4}&\cellcolor{color4}&\cellcolor{color4}&\cellcolor{color4}&\cellcolor{color4}&\cellcolor{color4}&\cellcolor{color4}&\cellcolor{color4}&\cellcolor{color4}&\cellcolor{color5}&\cellcolor{color0}&\cellcolor{color0}&\cellcolor{color0}&\cellcolor{color0}&\cellcolor{color0}&\cellcolor{color0}&\cellcolor{color0}&\cellcolor{color0}&\cellcolor{color1}\\
\cline{2-42}
&35 &\cellcolor{color1}&\cellcolor{color0}&\cellcolor{color0}&\cellcolor{color0}&\cellcolor{color0}&\cellcolor{color0}&\cellcolor{color0}&\cellcolor{color0}&\cellcolor{color0}&\cellcolor{color0}&\cellcolor{color0}&\cellcolor{color0}&\cellcolor{color0}&\cellcolor{color0}&\cellcolor{color0}&\cellcolor{color0}&\cellcolor{color0}&\cellcolor{color0}&\cellcolor{color0}&\cellcolor{color0}&\cellcolor{color0}&\cellcolor{color12}&\cellcolor{color13}&\cellcolor{color12}&\cellcolor{color13}&\cellcolor{color12}&\cellcolor{color13}&\cellcolor{color12}&\cellcolor{color13}&\cellcolor{color0}&\cellcolor{color0}&\cellcolor{color0}&\cellcolor{color0}&\cellcolor{color0}&\cellcolor{color0}&\cellcolor{color0}&\cellcolor{color0}&\cellcolor{color0}&\cellcolor{color0}&\cellcolor{color1}\\
\hline
\multirow{2}{*}{\tt \$550} &36 &\cellcolor{color1}&\cellcolor{color0}&\cellcolor{color0}&\cellcolor{color0}&\cellcolor{color0}&\cellcolor{color0}&\cellcolor{color0}&\cellcolor{color0}&\cellcolor{color0}&\cellcolor{color0}&\cellcolor{color0}&\cellcolor{color0}&\cellcolor{color0}&\cellcolor{color0}&\cellcolor{color0}&\cellcolor{color0}&\cellcolor{color0}&\cellcolor{color0}&\cellcolor{color0}&\cellcolor{color0}&\cellcolor{color0}&\cellcolor{color12}&\cellcolor{color13}&\cellcolor{color12}&\cellcolor{color13}&\cellcolor{color12}&\cellcolor{color13}&\cellcolor{color12}&\cellcolor{color13}&\cellcolor{color0}&\cellcolor{color0}&\cellcolor{color0}&\cellcolor{color0}&\cellcolor{color0}&\cellcolor{color0}&\cellcolor{color0}&\cellcolor{color0}&\cellcolor{color0}&\cellcolor{color0}&\cellcolor{color1}\\
\cline{2-42}
&37 &\cellcolor{color1}&\cellcolor{color0}&\cellcolor{color0}&\cellcolor{color0}&\cellcolor{color0}&\cellcolor{color0}&\cellcolor{color0}&\cellcolor{color0}&\cellcolor{color0}&\cellcolor{color0}&\cellcolor{color0}&\cellcolor{color0}&\cellcolor{color0}&\cellcolor{color0}&\cellcolor{color0}&\cellcolor{color0}&\cellcolor{color0}&\cellcolor{color0}&\cellcolor{color0}&\cellcolor{color0}&\cellcolor{color0}&\cellcolor{color0}&\cellcolor{color0}&\cellcolor{color0}&\cellcolor{color0}&\cellcolor{color0}&\cellcolor{color0}&\cellcolor{color0}&\cellcolor{color0}&\cellcolor{color0}&\cellcolor{color0}&\cellcolor{color0}&\cellcolor{color0}&\cellcolor{color0}&\cellcolor{color0}&\cellcolor{color0}&\cellcolor{color0}&\cellcolor{color0}&\cellcolor{color0}&\cellcolor{color1}\\
\hline
\multirow{2}{*}{\tt \$5D0} &38 &\cellcolor{color1}&\cellcolor{color0}&\cellcolor{color0}&\cellcolor{color0}&\cellcolor{color0}&\cellcolor{color0}&\cellcolor{color0}&\cellcolor{color0}&\cellcolor{color0}&\cellcolor{color0}&\cellcolor{color0}&\cellcolor{color0}&\cellcolor{color0}&\cellcolor{color0}&\cellcolor{color0}&\cellcolor{color0}&\cellcolor{color0}&\cellcolor{color0}&\cellcolor{color0}&\cellcolor{color0}&\cellcolor{color0}&\cellcolor{color0}&\cellcolor{color0}&\cellcolor{color0}&\cellcolor{color0}&\cellcolor{color0}&\cellcolor{color0}&\cellcolor{color0}&\cellcolor{color0}&\cellcolor{color0}&\cellcolor{color0}&\cellcolor{color0}&\cellcolor{color0}&\cellcolor{color0}&\cellcolor{color0}&\cellcolor{color0}&\cellcolor{color0}&\cellcolor{color0}&\cellcolor{color0}&\cellcolor{color1}\\
\cline{2-42}
&39 &\cellcolor{color1}&\cellcolor{color1}&\cellcolor{color1}&\cellcolor{color1}&\cellcolor{color1}&\cellcolor{color1}&\cellcolor{color1}&\cellcolor{color1}&\cellcolor{color1}&\cellcolor{color1}&\cellcolor{color1}&\cellcolor{color1}&\cellcolor{color1}&\cellcolor{color1}&\cellcolor{color1}&\cellcolor{color1}&\cellcolor{color1}&\cellcolor{color1}&\cellcolor{color1}&\cellcolor{color1}&\cellcolor{color1}&\cellcolor{color1}&\cellcolor{color1}&\cellcolor{color1}&\cellcolor{color1}&\cellcolor{color1}&\cellcolor{color1}&\cellcolor{color1}&\cellcolor{color1}&\cellcolor{color1}&\cellcolor{color1}&\cellcolor{color1}&\cellcolor{color1}&\cellcolor{color1}&\cellcolor{color1}&\cellcolor{color1}&\cellcolor{color1}&\cellcolor{color1}&\cellcolor{color1}&\cellcolor{color1}\\
\hline
\multirow{2}{*}{\tt \$650} &40 &\cellcolor{color0}&\cellcolor{color0}&\cellcolor{color0}&\cellcolor{color0}&\cellcolor{color0}&\cellcolor{color0}&\cellcolor{color0}&\cellcolor{color0}&\cellcolor{color0}&\cellcolor{color0}&\cellcolor{color0}&\cellcolor{color0}&\cellcolor{color0}&\cellcolor{color0}&\cellcolor{color0}&\cellcolor{color0}&\cellcolor{color0}&\cellcolor{color0}&\cellcolor{color0}&\cellcolor{color0}&\cellcolor{color0}&\cellcolor{color0}&\cellcolor{color0}&\cellcolor{color0}&\cellcolor{color0}&\cellcolor{color0}&\cellcolor{color0}&\cellcolor{color0}&\cellcolor{color0}&\cellcolor{color0}&\cellcolor{color0}&\cellcolor{color0}&\cellcolor{color0}&\cellcolor{color0}&\cellcolor{color0}&\cellcolor{color0}&\cellcolor{color0}&\cellcolor{color0}&\cellcolor{color0}&\cellcolor{color0}\\
\cline{2-42}
&41 &\cellcolor{color10}&\cellcolor{color10}&\cellcolor{color10}&\cellcolor{color10}&\cellcolor{color10}&\cellcolor{color10}&\cellcolor{color10}&\cellcolor{color10}&\cellcolor{color10}&\cellcolor{color10}&\cellcolor{color10}&\cellcolor{color10}&\cellcolor{color10}&\cellcolor{color10}&\cellcolor{color10}&\cellcolor{color10}&\cellcolor{color10}&\cellcolor{color10}&\cellcolor{color10}&\cellcolor{color10}&\cellcolor{color10}&\cellcolor{color10}&\cellcolor{color10}&\cellcolor{color10}&\cellcolor{color10}&\cellcolor{color10}&\cellcolor{color10}&\cellcolor{color10}&\cellcolor{color10}&\cellcolor{color10}&\cellcolor{color10}&\cellcolor{color10}&\cellcolor{color10}&\cellcolor{color10}&\cellcolor{color10}&\cellcolor{color10}&\cellcolor{color10}&\cellcolor{color10}&\cellcolor{color10}&\cellcolor{color10}\\
\hline
\multirow{2}{*}{\tt \$6D0} &42 &\cellcolor{color12}&\cellcolor{color15}&\cellcolor{color1}&\cellcolor{color4}&\cellcolor{color9}&\cellcolor{color14}&\cellcolor{color7}&\cellcolor{color0}&\cellcolor{color12}&\cellcolor{color9}&\cellcolor{color2}&\cellcolor{color1}&\cellcolor{color2}&\cellcolor{color14}&\cellcolor{color11}&\cellcolor{color2}&\cellcolor{color7}&\cellcolor{color0}&\cellcolor{color0}&\cellcolor{color0}&\cellcolor{color0}&\cellcolor{color0}&\cellcolor{color0}&\cellcolor{color0}&\cellcolor{color0}&\cellcolor{color0}&\cellcolor{color0}&\cellcolor{color0}&\cellcolor{color0}&\cellcolor{color0}&\cellcolor{color0}&\cellcolor{color0}&\cellcolor{color0}&\cellcolor{color0}&\cellcolor{color0}&\cellcolor{color0}&\cellcolor{color0}&\cellcolor{color0}&\cellcolor{color0}&\cellcolor{color0}\\
\cline{2-42}
&43 &\cellcolor{color12}&\cellcolor{color12}&\cellcolor{color12}&\cellcolor{color12}&\cellcolor{color12}&\cellcolor{color12}&\cellcolor{color12}&\cellcolor{color10}&\cellcolor{color12}&\cellcolor{color13}&\cellcolor{color13}&\cellcolor{color12}&\cellcolor{color11}&\cellcolor{color10}&\cellcolor{color12}&\cellcolor{color13}&\cellcolor{color13}&\cellcolor{color10}&\cellcolor{color10}&\cellcolor{color10}&\cellcolor{color10}&\cellcolor{color10}&\cellcolor{color10}&\cellcolor{color10}&\cellcolor{color10}&\cellcolor{color10}&\cellcolor{color10}&\cellcolor{color10}&\cellcolor{color10}&\cellcolor{color10}&\cellcolor{color10}&\cellcolor{color10}&\cellcolor{color10}&\cellcolor{color10}&\cellcolor{color10}&\cellcolor{color10}&\cellcolor{color10}&\cellcolor{color10}&\cellcolor{color10}&\cellcolor{color10}\\
\hline
\multirow{2}{*}{\tt \$750} &44 &\cellcolor{color0}&\cellcolor{color0}&\cellcolor{color0}&\cellcolor{color0}&\cellcolor{color0}&\cellcolor{color0}&\cellcolor{color0}&\cellcolor{color0}&\cellcolor{color0}&\cellcolor{color0}&\cellcolor{color0}&\cellcolor{color0}&\cellcolor{color0}&\cellcolor{color0}&\cellcolor{color0}&\cellcolor{color0}&\cellcolor{color0}&\cellcolor{color0}&\cellcolor{color0}&\cellcolor{color0}&\cellcolor{color0}&\cellcolor{color0}&\cellcolor{color0}&\cellcolor{color0}&\cellcolor{color0}&\cellcolor{color0}&\cellcolor{color0}&\cellcolor{color0}&\cellcolor{color0}&\cellcolor{color0}&\cellcolor{color0}&\cellcolor{color0}&\cellcolor{color0}&\cellcolor{color0}&\cellcolor{color0}&\cellcolor{color0}&\cellcolor{color0}&\cellcolor{color0}&\cellcolor{color0}&\cellcolor{color0}\\
\cline{2-42}
&45 &\cellcolor{color10}&\cellcolor{color10}&\cellcolor{color10}&\cellcolor{color10}&\cellcolor{color10}&\cellcolor{color10}&\cellcolor{color10}&\cellcolor{color10}&\cellcolor{color10}&\cellcolor{color10}&\cellcolor{color10}&\cellcolor{color10}&\cellcolor{color10}&\cellcolor{color10}&\cellcolor{color10}&\cellcolor{color10}&\cellcolor{color10}&\cellcolor{color10}&\cellcolor{color10}&\cellcolor{color10}&\cellcolor{color10}&\cellcolor{color10}&\cellcolor{color10}&\cellcolor{color10}&\cellcolor{color10}&\cellcolor{color10}&\cellcolor{color10}&\cellcolor{color10}&\cellcolor{color10}&\cellcolor{color10}&\cellcolor{color10}&\cellcolor{color10}&\cellcolor{color10}&\cellcolor{color10}&\cellcolor{color10}&\cellcolor{color10}&\cellcolor{color10}&\cellcolor{color10}&\cellcolor{color10}&\cellcolor{color10}\\
\hline
\multirow{2}{*}{\tt \$7D0} &46 &\cellcolor{color0}&\cellcolor{color0}&\cellcolor{color0}&\cellcolor{color0}&\cellcolor{color0}&\cellcolor{color0}&\cellcolor{color0}&\cellcolor{color0}&\cellcolor{color0}&\cellcolor{color0}&\cellcolor{color0}&\cellcolor{color0}&\cellcolor{color0}&\cellcolor{color0}&\cellcolor{color0}&\cellcolor{color0}&\cellcolor{color0}&\cellcolor{color0}&\cellcolor{color0}&\cellcolor{color0}&\cellcolor{color0}&\cellcolor{color0}&\cellcolor{color0}&\cellcolor{color0}&\cellcolor{color0}&\cellcolor{color0}&\cellcolor{color0}&\cellcolor{color0}&\cellcolor{color0}&\cellcolor{color0}&\cellcolor{color0}&\cellcolor{color0}&\cellcolor{color0}&\cellcolor{color0}&\cellcolor{color0}&\cellcolor{color0}&\cellcolor{color0}&\cellcolor{color0}&\cellcolor{color0}&\cellcolor{color0}\\
\cline{2-42}
&47 &\cellcolor{color10}&\cellcolor{color10}&\cellcolor{color10}&\cellcolor{color10}&\cellcolor{color10}&\cellcolor{color10}&\cellcolor{color10}&\cellcolor{color10}&\cellcolor{color10}&\cellcolor{color10}&\cellcolor{color10}&\cellcolor{color10}&\cellcolor{color10}&\cellcolor{color10}&\cellcolor{color10}&\cellcolor{color10}&\cellcolor{color10}&\cellcolor{color10}&\cellcolor{color10}&\cellcolor{color10}&\cellcolor{color10}&\cellcolor{color10}&\cellcolor{color10}&\cellcolor{color10}&\cellcolor{color10}&\cellcolor{color10}&\cellcolor{color10}&\cellcolor{color10}&\cellcolor{color10}&\cellcolor{color10}&\cellcolor{color10}&\cellcolor{color10}&\cellcolor{color10}&\cellcolor{color10}&\cellcolor{color10}&\cellcolor{color10}&\cellcolor{color10}&\cellcolor{color10}&\cellcolor{color10}&\cellcolor{color10}\\
\hline
\end{tabular}
}
\end{table*}
\end{tiny}
\begin{tiny}



\begin{table*}
\caption{The same image from Table 1, but viewed with linear addresses.
Note the screen ``holes'' which pad every third line to a 128 byte boundary.
This unused memory can be used by I/O cards. \label{table:linear}}
\centering
\begin{tabular}{|l|l|c|c|c|c|c|c|c|c|c|c|c|c|c|c|c|c|c|c|c|c|c|c|c|c|c|c|c|c|c|c|c|c|c|c|c|c|c|c|c|c|c|c|c|c|c|c|c|c|}
\hline
\multirow{2}{*}{\tt \$400} &0 &\cellcolor{color0}&\cellcolor{color0}&\cellcolor{color0}&\cellcolor{color0}&\cellcolor{color0}&\cellcolor{color0}&\cellcolor{color0}&\cellcolor{color0}&\cellcolor{color0}&\cellcolor{color0}&\cellcolor{color0}&\cellcolor{color0}&\cellcolor{color0}&\cellcolor{color0}&\cellcolor{color0}&\cellcolor{color0}&\cellcolor{color0}&\cellcolor{color0}&\cellcolor{color0}&\cellcolor{color0}&\cellcolor{color0}&\cellcolor{color0}&\cellcolor{color0}&\cellcolor{color0}&\cellcolor{color0}&\cellcolor{color0}&\cellcolor{color0}&\cellcolor{color0}&\cellcolor{color0}&\cellcolor{color0}&\cellcolor{color0}&\cellcolor{color0}&\cellcolor{color0}&\cellcolor{color0}&\cellcolor{color0}&\cellcolor{color0}&\cellcolor{color0}&\cellcolor{color0}&\cellcolor{color0}&\cellcolor{color0}\\
\cline{2-42}
&1 &\cellcolor{color0}&\cellcolor{color0}&\cellcolor{color0}&\cellcolor{color0}&\cellcolor{color0}&\cellcolor{color0}&\cellcolor{color0}&\cellcolor{color0}&\cellcolor{color0}&\cellcolor{color0}&\cellcolor{color0}&\cellcolor{color0}&\cellcolor{color0}&\cellcolor{color0}&\cellcolor{color1}&\cellcolor{color1}&\cellcolor{color1}&\cellcolor{color4}&\cellcolor{color2}&\cellcolor{color2}&\cellcolor{color2}&\cellcolor{color4}&\cellcolor{color2}&\cellcolor{color2}&\cellcolor{color2}&\cellcolor{color0}&\cellcolor{color0}&\cellcolor{color0}&\cellcolor{color0}&\cellcolor{color0}&\cellcolor{color0}&\cellcolor{color0}&\cellcolor{color0}&\cellcolor{color0}&\cellcolor{color0}&\cellcolor{color0}&\cellcolor{color0}&\cellcolor{color0}&\cellcolor{color0}&\cellcolor{color0}\\
\hline
\multirow{2}{*}{\tt \$428} &2 &\cellcolor{color0}&\cellcolor{color1}&\cellcolor{color0}&\cellcolor{color2}&\cellcolor{color2}&\cellcolor{color2}&\cellcolor{color0}&\cellcolor{color2}&\cellcolor{color0}&\cellcolor{color2}&\cellcolor{color0}&\cellcolor{color2}&\cellcolor{color0}&\cellcolor{color2}&\cellcolor{color0}&\cellcolor{color0}&\cellcolor{color0}&\cellcolor{color0}&\cellcolor{color3}&\cellcolor{color0}&\cellcolor{color0}&\cellcolor{color3}&\cellcolor{color3}&\cellcolor{color3}&\cellcolor{color0}&\cellcolor{color3}&\cellcolor{color0}&\cellcolor{color0}&\cellcolor{color3}&\cellcolor{color0}&\cellcolor{color3}&\cellcolor{color3}&\cellcolor{color3}&\cellcolor{color0}&\cellcolor{color3}&\cellcolor{color3}&\cellcolor{color3}&\cellcolor{color0}&\cellcolor{color1}&\cellcolor{color0}\\
\cline{2-42}
&3 &\cellcolor{color0}&\cellcolor{color1}&\cellcolor{color0}&\cellcolor{color0}&\cellcolor{color2}&\cellcolor{color2}&\cellcolor{color0}&\cellcolor{color2}&\cellcolor{color0}&\cellcolor{color2}&\cellcolor{color0}&\cellcolor{color2}&\cellcolor{color0}&\cellcolor{color2}&\cellcolor{color0}&\cellcolor{color0}&\cellcolor{color0}&\cellcolor{color0}&\cellcolor{color3}&\cellcolor{color0}&\cellcolor{color0}&\cellcolor{color3}&\cellcolor{color3}&\cellcolor{color3}&\cellcolor{color0}&\cellcolor{color3}&\cellcolor{color0}&\cellcolor{color0}&\cellcolor{color3}&\cellcolor{color0}&\cellcolor{color3}&\cellcolor{color3}&\cellcolor{color3}&\cellcolor{color0}&\cellcolor{color3}&\cellcolor{color3}&\cellcolor{color3}&\cellcolor{color0}&\cellcolor{color1}&\cellcolor{color0}\\
\hline
\multirow{2}{*}{\tt \$450} &4 &\cellcolor{color1}&\cellcolor{color0}&\cellcolor{color0}&\cellcolor{color0}&\cellcolor{color0}&\cellcolor{color0}&\cellcolor{color0}&\cellcolor{color0}&\cellcolor{color0}&\cellcolor{color4}&\cellcolor{color4}&\cellcolor{color4}&\cellcolor{color4}&\cellcolor{color4}&\cellcolor{color4}&\cellcolor{color4}&\cellcolor{color4}&\cellcolor{color4}&\cellcolor{color4}&\cellcolor{color4}&\cellcolor{color4}&\cellcolor{color4}&\cellcolor{color4}&\cellcolor{color4}&\cellcolor{color4}&\cellcolor{color4}&\cellcolor{color4}&\cellcolor{color4}&\cellcolor{color4}&\cellcolor{color4}&\cellcolor{color7}&\cellcolor{color0}&\cellcolor{color0}&\cellcolor{color0}&\cellcolor{color0}&\cellcolor{color0}&\cellcolor{color0}&\cellcolor{color0}&\cellcolor{color0}&\cellcolor{color1}&\cellcolor{color12}&\cellcolor{color15}&\cellcolor{color0}&\cellcolor{color0}&\cellcolor{color15}&\cellcolor{color15}&\cellcolor{color8}&\cellcolor{color0}\\
\cline{2-42}
&5 &\cellcolor{color1}&\cellcolor{color0}&\cellcolor{color0}&\cellcolor{color0}&\cellcolor{color0}&\cellcolor{color0}&\cellcolor{color0}&\cellcolor{color0}&\cellcolor{color0}&\cellcolor{color4}&\cellcolor{color4}&\cellcolor{color4}&\cellcolor{color4}&\cellcolor{color4}&\cellcolor{color4}&\cellcolor{color4}&\cellcolor{color4}&\cellcolor{color4}&\cellcolor{color4}&\cellcolor{color4}&\cellcolor{color4}&\cellcolor{color4}&\cellcolor{color4}&\cellcolor{color4}&\cellcolor{color4}&\cellcolor{color4}&\cellcolor{color4}&\cellcolor{color4}&\cellcolor{color4}&\cellcolor{color4}&\cellcolor{color5}&\cellcolor{color0}&\cellcolor{color0}&\cellcolor{color0}&\cellcolor{color0}&\cellcolor{color0}&\cellcolor{color0}&\cellcolor{color0}&\cellcolor{color0}&\cellcolor{color1}&\cellcolor{color0}&\cellcolor{color15}&\cellcolor{color0}&\cellcolor{color0}&\cellcolor{color15}&\cellcolor{color15}&\cellcolor{color1}&\cellcolor{color0}\\
\hline
\multirow{2}{*}{\tt \$480} &6 &\cellcolor{color0}&\cellcolor{color0}&\cellcolor{color0}&\cellcolor{color0}&\cellcolor{color0}&\cellcolor{color0}&\cellcolor{color0}&\cellcolor{color0}&\cellcolor{color1}&\cellcolor{color1}&\cellcolor{color1}&\cellcolor{color1}&\cellcolor{color1}&\cellcolor{color0}&\cellcolor{color1}&\cellcolor{color1}&\cellcolor{color1}&\cellcolor{color4}&\cellcolor{color2}&\cellcolor{color2}&\cellcolor{color2}&\cellcolor{color4}&\cellcolor{color2}&\cellcolor{color2}&\cellcolor{color2}&\cellcolor{color0}&\cellcolor{color1}&\cellcolor{color1}&\cellcolor{color1}&\cellcolor{color1}&\cellcolor{color1}&\cellcolor{color0}&\cellcolor{color0}&\cellcolor{color0}&\cellcolor{color0}&\cellcolor{color0}&\cellcolor{color0}&\cellcolor{color0}&\cellcolor{color0}&\cellcolor{color0}\\
\cline{2-42}
&7 &\cellcolor{color0}&\cellcolor{color0}&\cellcolor{color0}&\cellcolor{color0}&\cellcolor{color0}&\cellcolor{color1}&\cellcolor{color1}&\cellcolor{color1}&\cellcolor{color1}&\cellcolor{color9}&\cellcolor{color9}&\cellcolor{color9}&\cellcolor{color9}&\cellcolor{color0}&\cellcolor{color1}&\cellcolor{color1}&\cellcolor{color1}&\cellcolor{color4}&\cellcolor{color2}&\cellcolor{color2}&\cellcolor{color2}&\cellcolor{color4}&\cellcolor{color2}&\cellcolor{color2}&\cellcolor{color2}&\cellcolor{color0}&\cellcolor{color9}&\cellcolor{color9}&\cellcolor{color9}&\cellcolor{color9}&\cellcolor{color1}&\cellcolor{color1}&\cellcolor{color1}&\cellcolor{color1}&\cellcolor{color0}&\cellcolor{color0}&\cellcolor{color0}&\cellcolor{color0}&\cellcolor{color0}&\cellcolor{color0}\\
\hline
\multirow{2}{*}{\tt \$4A8} &8 &\cellcolor{color0}&\cellcolor{color1}&\cellcolor{color0}&\cellcolor{color0}&\cellcolor{color0}&\cellcolor{color0}&\cellcolor{color0}&\cellcolor{color0}&\cellcolor{color0}&\cellcolor{color0}&\cellcolor{color0}&\cellcolor{color0}&\cellcolor{color0}&\cellcolor{color0}&\cellcolor{color0}&\cellcolor{color0}&\cellcolor{color0}&\cellcolor{color0}&\cellcolor{color0}&\cellcolor{color0}&\cellcolor{color0}&\cellcolor{color0}&\cellcolor{color0}&\cellcolor{color0}&\cellcolor{color0}&\cellcolor{color0}&\cellcolor{color0}&\cellcolor{color0}&\cellcolor{color0}&\cellcolor{color0}&\cellcolor{color0}&\cellcolor{color0}&\cellcolor{color0}&\cellcolor{color0}&\cellcolor{color0}&\cellcolor{color0}&\cellcolor{color0}&\cellcolor{color0}&\cellcolor{color1}&\cellcolor{color0}\\
\cline{2-42}
&9 &\cellcolor{color1}&\cellcolor{color1}&\cellcolor{color0}&\cellcolor{color0}&\cellcolor{color0}&\cellcolor{color0}&\cellcolor{color0}&\cellcolor{color0}&\cellcolor{color0}&\cellcolor{color0}&\cellcolor{color0}&\cellcolor{color0}&\cellcolor{color0}&\cellcolor{color0}&\cellcolor{color0}&\cellcolor{color0}&\cellcolor{color0}&\cellcolor{color0}&\cellcolor{color0}&\cellcolor{color0}&\cellcolor{color0}&\cellcolor{color0}&\cellcolor{color0}&\cellcolor{color0}&\cellcolor{color0}&\cellcolor{color0}&\cellcolor{color0}&\cellcolor{color0}&\cellcolor{color0}&\cellcolor{color0}&\cellcolor{color0}&\cellcolor{color0}&\cellcolor{color0}&\cellcolor{color0}&\cellcolor{color0}&\cellcolor{color0}&\cellcolor{color0}&\cellcolor{color0}&\cellcolor{color1}&\cellcolor{color1}\\
\hline
\multirow{2}{*}{\tt \$4D0} &10 &\cellcolor{color1}&\cellcolor{color0}&\cellcolor{color0}&\cellcolor{color0}&\cellcolor{color0}&\cellcolor{color0}&\cellcolor{color0}&\cellcolor{color0}&\cellcolor{color0}&\cellcolor{color5}&\cellcolor{color4}&\cellcolor{color4}&\cellcolor{color4}&\cellcolor{color4}&\cellcolor{color4}&\cellcolor{color4}&\cellcolor{color4}&\cellcolor{color4}&\cellcolor{color4}&\cellcolor{color4}&\cellcolor{color4}&\cellcolor{color4}&\cellcolor{color4}&\cellcolor{color4}&\cellcolor{color4}&\cellcolor{color4}&\cellcolor{color4}&\cellcolor{color4}&\cellcolor{color4}&\cellcolor{color4}&\cellcolor{color5}&\cellcolor{color0}&\cellcolor{color0}&\cellcolor{color0}&\cellcolor{color0}&\cellcolor{color0}&\cellcolor{color0}&\cellcolor{color0}&\cellcolor{color0}&\cellcolor{color1}&\cellcolor{color4}&\cellcolor{color15}&\cellcolor{color0}&\cellcolor{color15}&\cellcolor{color15}&\cellcolor{color15}&\cellcolor{color0}&\cellcolor{color0}\\
\cline{2-42}
&11 &\cellcolor{color1}&\cellcolor{color0}&\cellcolor{color0}&\cellcolor{color0}&\cellcolor{color0}&\cellcolor{color0}&\cellcolor{color0}&\cellcolor{color0}&\cellcolor{color0}&\cellcolor{color0}&\cellcolor{color0}&\cellcolor{color0}&\cellcolor{color0}&\cellcolor{color0}&\cellcolor{color0}&\cellcolor{color0}&\cellcolor{color0}&\cellcolor{color0}&\cellcolor{color0}&\cellcolor{color0}&\cellcolor{color0}&\cellcolor{color12}&\cellcolor{color13}&\cellcolor{color12}&\cellcolor{color13}&\cellcolor{color12}&\cellcolor{color13}&\cellcolor{color12}&\cellcolor{color13}&\cellcolor{color0}&\cellcolor{color0}&\cellcolor{color0}&\cellcolor{color0}&\cellcolor{color0}&\cellcolor{color0}&\cellcolor{color0}&\cellcolor{color0}&\cellcolor{color0}&\cellcolor{color0}&\cellcolor{color1}&\cellcolor{color0}&\cellcolor{color15}&\cellcolor{color0}&\cellcolor{color15}&\cellcolor{color15}&\cellcolor{color15}&\cellcolor{color0}&\cellcolor{color0}\\
\hline
\multirow{2}{*}{\tt \$500} &12 &\cellcolor{color0}&\cellcolor{color0}&\cellcolor{color0}&\cellcolor{color1}&\cellcolor{color1}&\cellcolor{color1}&\cellcolor{color9}&\cellcolor{color9}&\cellcolor{color9}&\cellcolor{color9}&\cellcolor{color13}&\cellcolor{color13}&\cellcolor{color13}&\cellcolor{color0}&\cellcolor{color0}&\cellcolor{color1}&\cellcolor{color4}&\cellcolor{color4}&\cellcolor{color4}&\cellcolor{color2}&\cellcolor{color4}&\cellcolor{color4}&\cellcolor{color4}&\cellcolor{color2}&\cellcolor{color0}&\cellcolor{color0}&\cellcolor{color13}&\cellcolor{color13}&\cellcolor{color13}&\cellcolor{color9}&\cellcolor{color9}&\cellcolor{color9}&\cellcolor{color9}&\cellcolor{color1}&\cellcolor{color1}&\cellcolor{color1}&\cellcolor{color1}&\cellcolor{color0}&\cellcolor{color0}&\cellcolor{color0}\\
\cline{2-42}
&13 &\cellcolor{color0}&\cellcolor{color0}&\cellcolor{color1}&\cellcolor{color1}&\cellcolor{color9}&\cellcolor{color9}&\cellcolor{color9}&\cellcolor{color13}&\cellcolor{color13}&\cellcolor{color13}&\cellcolor{color13}&\cellcolor{color12}&\cellcolor{color12}&\cellcolor{color12}&\cellcolor{color0}&\cellcolor{color1}&\cellcolor{color4}&\cellcolor{color4}&\cellcolor{color4}&\cellcolor{color2}&\cellcolor{color4}&\cellcolor{color4}&\cellcolor{color4}&\cellcolor{color2}&\cellcolor{color0}&\cellcolor{color12}&\cellcolor{color12}&\cellcolor{color12}&\cellcolor{color13}&\cellcolor{color13}&\cellcolor{color13}&\cellcolor{color13}&\cellcolor{color9}&\cellcolor{color9}&\cellcolor{color9}&\cellcolor{color9}&\cellcolor{color1}&\cellcolor{color1}&\cellcolor{color0}&\cellcolor{color0}\\
\hline
\multirow{2}{*}{\tt \$528} &14 &\cellcolor{color1}&\cellcolor{color0}&\cellcolor{color0}&\cellcolor{color0}&\cellcolor{color0}&\cellcolor{color0}&\cellcolor{color0}&\cellcolor{color0}&\cellcolor{color0}&\cellcolor{color0}&\cellcolor{color0}&\cellcolor{color0}&\cellcolor{color0}&\cellcolor{color0}&\cellcolor{color0}&\cellcolor{color0}&\cellcolor{color0}&\cellcolor{color0}&\cellcolor{color0}&\cellcolor{color0}&\cellcolor{color0}&\cellcolor{color0}&\cellcolor{color0}&\cellcolor{color0}&\cellcolor{color0}&\cellcolor{color0}&\cellcolor{color0}&\cellcolor{color0}&\cellcolor{color0}&\cellcolor{color0}&\cellcolor{color0}&\cellcolor{color0}&\cellcolor{color0}&\cellcolor{color0}&\cellcolor{color0}&\cellcolor{color0}&\cellcolor{color0}&\cellcolor{color0}&\cellcolor{color0}&\cellcolor{color1}\\
\cline{2-42}
&15 &\cellcolor{color1}&\cellcolor{color0}&\cellcolor{color0}&\cellcolor{color0}&\cellcolor{color0}&\cellcolor{color0}&\cellcolor{color0}&\cellcolor{color0}&\cellcolor{color0}&\cellcolor{color4}&\cellcolor{color4}&\cellcolor{color4}&\cellcolor{color4}&\cellcolor{color4}&\cellcolor{color4}&\cellcolor{color4}&\cellcolor{color4}&\cellcolor{color4}&\cellcolor{color4}&\cellcolor{color4}&\cellcolor{color4}&\cellcolor{color4}&\cellcolor{color4}&\cellcolor{color4}&\cellcolor{color4}&\cellcolor{color4}&\cellcolor{color4}&\cellcolor{color4}&\cellcolor{color4}&\cellcolor{color4}&\cellcolor{color0}&\cellcolor{color0}&\cellcolor{color0}&\cellcolor{color0}&\cellcolor{color0}&\cellcolor{color0}&\cellcolor{color0}&\cellcolor{color0}&\cellcolor{color0}&\cellcolor{color1}\\
\hline
\multirow{2}{*}{\tt \$550} &16 &\cellcolor{color1}&\cellcolor{color0}&\cellcolor{color0}&\cellcolor{color0}&\cellcolor{color0}&\cellcolor{color0}&\cellcolor{color0}&\cellcolor{color0}&\cellcolor{color0}&\cellcolor{color0}&\cellcolor{color0}&\cellcolor{color0}&\cellcolor{color0}&\cellcolor{color0}&\cellcolor{color0}&\cellcolor{color0}&\cellcolor{color0}&\cellcolor{color0}&\cellcolor{color0}&\cellcolor{color0}&\cellcolor{color0}&\cellcolor{color12}&\cellcolor{color13}&\cellcolor{color12}&\cellcolor{color13}&\cellcolor{color12}&\cellcolor{color13}&\cellcolor{color12}&\cellcolor{color13}&\cellcolor{color0}&\cellcolor{color0}&\cellcolor{color0}&\cellcolor{color0}&\cellcolor{color0}&\cellcolor{color0}&\cellcolor{color0}&\cellcolor{color0}&\cellcolor{color0}&\cellcolor{color0}&\cellcolor{color1}&\cellcolor{color3}&\cellcolor{color15}&\cellcolor{color0}&\cellcolor{color0}&\cellcolor{color15}&\cellcolor{color15}&\cellcolor{color0}&\cellcolor{color0}\\
\cline{2-42}
&17 &\cellcolor{color1}&\cellcolor{color0}&\cellcolor{color0}&\cellcolor{color0}&\cellcolor{color0}&\cellcolor{color0}&\cellcolor{color0}&\cellcolor{color0}&\cellcolor{color0}&\cellcolor{color0}&\cellcolor{color0}&\cellcolor{color0}&\cellcolor{color0}&\cellcolor{color0}&\cellcolor{color0}&\cellcolor{color0}&\cellcolor{color0}&\cellcolor{color0}&\cellcolor{color0}&\cellcolor{color0}&\cellcolor{color0}&\cellcolor{color0}&\cellcolor{color0}&\cellcolor{color0}&\cellcolor{color0}&\cellcolor{color0}&\cellcolor{color0}&\cellcolor{color0}&\cellcolor{color0}&\cellcolor{color0}&\cellcolor{color0}&\cellcolor{color0}&\cellcolor{color0}&\cellcolor{color0}&\cellcolor{color0}&\cellcolor{color0}&\cellcolor{color0}&\cellcolor{color0}&\cellcolor{color0}&\cellcolor{color1}&\cellcolor{color2}&\cellcolor{color15}&\cellcolor{color0}&\cellcolor{color0}&\cellcolor{color15}&\cellcolor{color15}&\cellcolor{color0}&\cellcolor{color0}\\
\hline
\multirow{2}{*}{\tt \$580} &18 &\cellcolor{color0}&\cellcolor{color0}&\cellcolor{color1}&\cellcolor{color9}&\cellcolor{color9}&\cellcolor{color13}&\cellcolor{color13}&\cellcolor{color13}&\cellcolor{color12}&\cellcolor{color12}&\cellcolor{color12}&\cellcolor{color6}&\cellcolor{color6}&\cellcolor{color6}&\cellcolor{color0}&\cellcolor{color1}&\cellcolor{color4}&\cellcolor{color4}&\cellcolor{color4}&\cellcolor{color2}&\cellcolor{color4}&\cellcolor{color4}&\cellcolor{color4}&\cellcolor{color2}&\cellcolor{color0}&\cellcolor{color6}&\cellcolor{color6}&\cellcolor{color6}&\cellcolor{color12}&\cellcolor{color12}&\cellcolor{color12}&\cellcolor{color13}&\cellcolor{color13}&\cellcolor{color13}&\cellcolor{color13}&\cellcolor{color9}&\cellcolor{color9}&\cellcolor{color1}&\cellcolor{color0}&\cellcolor{color0}\\
\cline{2-42}
&19 &\cellcolor{color0}&\cellcolor{color1}&\cellcolor{color9}&\cellcolor{color13}&\cellcolor{color13}&\cellcolor{color13}&\cellcolor{color12}&\cellcolor{color12}&\cellcolor{color6}&\cellcolor{color6}&\cellcolor{color6}&\cellcolor{color0}&\cellcolor{color0}&\cellcolor{color0}&\cellcolor{color0}&\cellcolor{color0}&\cellcolor{color0}&\cellcolor{color0}&\cellcolor{color0}&\cellcolor{color0}&\cellcolor{color0}&\cellcolor{color0}&\cellcolor{color0}&\cellcolor{color0}&\cellcolor{color0}&\cellcolor{color0}&\cellcolor{color0}&\cellcolor{color0}&\cellcolor{color6}&\cellcolor{color6}&\cellcolor{color6}&\cellcolor{color12}&\cellcolor{color12}&\cellcolor{color12}&\cellcolor{color13}&\cellcolor{color13}&\cellcolor{color13}&\cellcolor{color9}&\cellcolor{color1}&\cellcolor{color0}\\
\hline
\multirow{2}{*}{\tt \$5A8} &20 &\cellcolor{color1}&\cellcolor{color0}&\cellcolor{color0}&\cellcolor{color0}&\cellcolor{color0}&\cellcolor{color0}&\cellcolor{color0}&\cellcolor{color0}&\cellcolor{color0}&\cellcolor{color4}&\cellcolor{color5}&\cellcolor{color5}&\cellcolor{color5}&\cellcolor{color4}&\cellcolor{color5}&\cellcolor{color5}&\cellcolor{color5}&\cellcolor{color5}&\cellcolor{color5}&\cellcolor{color4}&\cellcolor{color5}&\cellcolor{color5}&\cellcolor{color5}&\cellcolor{color4}&\cellcolor{color4}&\cellcolor{color4}&\cellcolor{color14}&\cellcolor{color4}&\cellcolor{color14}&\cellcolor{color4}&\cellcolor{color0}&\cellcolor{color0}&\cellcolor{color0}&\cellcolor{color0}&\cellcolor{color0}&\cellcolor{color0}&\cellcolor{color0}&\cellcolor{color0}&\cellcolor{color0}&\cellcolor{color1}\\
\cline{2-42}
&21 &\cellcolor{color1}&\cellcolor{color0}&\cellcolor{color0}&\cellcolor{color0}&\cellcolor{color0}&\cellcolor{color12}&\cellcolor{color0}&\cellcolor{color0}&\cellcolor{color0}&\cellcolor{color4}&\cellcolor{color5}&\cellcolor{color5}&\cellcolor{color5}&\cellcolor{color4}&\cellcolor{color5}&\cellcolor{color5}&\cellcolor{color5}&\cellcolor{color5}&\cellcolor{color5}&\cellcolor{color4}&\cellcolor{color5}&\cellcolor{color4}&\cellcolor{color5}&\cellcolor{color4}&\cellcolor{color9}&\cellcolor{color4}&\cellcolor{color4}&\cellcolor{color4}&\cellcolor{color4}&\cellcolor{color4}&\cellcolor{color0}&\cellcolor{color0}&\cellcolor{color0}&\cellcolor{color0}&\cellcolor{color0}&\cellcolor{color0}&\cellcolor{color12}&\cellcolor{color0}&\cellcolor{color0}&\cellcolor{color1}\\
\hline
\multirow{2}{*}{\tt \$5D0} &22 &\cellcolor{color1}&\cellcolor{color0}&\cellcolor{color0}&\cellcolor{color0}&\cellcolor{color0}&\cellcolor{color0}&\cellcolor{color0}&\cellcolor{color0}&\cellcolor{color0}&\cellcolor{color0}&\cellcolor{color0}&\cellcolor{color0}&\cellcolor{color0}&\cellcolor{color0}&\cellcolor{color0}&\cellcolor{color0}&\cellcolor{color0}&\cellcolor{color0}&\cellcolor{color0}&\cellcolor{color0}&\cellcolor{color0}&\cellcolor{color0}&\cellcolor{color0}&\cellcolor{color0}&\cellcolor{color0}&\cellcolor{color0}&\cellcolor{color0}&\cellcolor{color0}&\cellcolor{color0}&\cellcolor{color0}&\cellcolor{color0}&\cellcolor{color0}&\cellcolor{color0}&\cellcolor{color0}&\cellcolor{color0}&\cellcolor{color0}&\cellcolor{color0}&\cellcolor{color0}&\cellcolor{color0}&\cellcolor{color1}&\cellcolor{color0}&\cellcolor{color15}&\cellcolor{color0}&\cellcolor{color7}&\cellcolor{color15}&\cellcolor{color15}&\cellcolor{color0}&\cellcolor{color0}\\
\cline{2-42}
&23 &\cellcolor{color1}&\cellcolor{color1}&\cellcolor{color1}&\cellcolor{color1}&\cellcolor{color1}&\cellcolor{color1}&\cellcolor{color1}&\cellcolor{color1}&\cellcolor{color1}&\cellcolor{color1}&\cellcolor{color1}&\cellcolor{color1}&\cellcolor{color1}&\cellcolor{color1}&\cellcolor{color1}&\cellcolor{color1}&\cellcolor{color1}&\cellcolor{color1}&\cellcolor{color1}&\cellcolor{color1}&\cellcolor{color1}&\cellcolor{color1}&\cellcolor{color1}&\cellcolor{color1}&\cellcolor{color1}&\cellcolor{color1}&\cellcolor{color1}&\cellcolor{color1}&\cellcolor{color1}&\cellcolor{color1}&\cellcolor{color1}&\cellcolor{color1}&\cellcolor{color1}&\cellcolor{color1}&\cellcolor{color1}&\cellcolor{color1}&\cellcolor{color1}&\cellcolor{color1}&\cellcolor{color1}&\cellcolor{color1}&\cellcolor{color6}&\cellcolor{color15}&\cellcolor{color0}&\cellcolor{color1}&\cellcolor{color15}&\cellcolor{color15}&\cellcolor{color0}&\cellcolor{color0}\\
\hline
\multirow{2}{*}{\tt \$600} &24 &\cellcolor{color0}&\cellcolor{color1}&\cellcolor{color0}&\cellcolor{color0}&\cellcolor{color0}&\cellcolor{color0}&\cellcolor{color0}&\cellcolor{color0}&\cellcolor{color0}&\cellcolor{color0}&\cellcolor{color0}&\cellcolor{color0}&\cellcolor{color0}&\cellcolor{color0}&\cellcolor{color0}&\cellcolor{color0}&\cellcolor{color0}&\cellcolor{color0}&\cellcolor{color0}&\cellcolor{color0}&\cellcolor{color0}&\cellcolor{color0}&\cellcolor{color0}&\cellcolor{color0}&\cellcolor{color0}&\cellcolor{color0}&\cellcolor{color0}&\cellcolor{color0}&\cellcolor{color0}&\cellcolor{color0}&\cellcolor{color0}&\cellcolor{color0}&\cellcolor{color0}&\cellcolor{color0}&\cellcolor{color0}&\cellcolor{color0}&\cellcolor{color0}&\cellcolor{color0}&\cellcolor{color1}&\cellcolor{color0}\\
\cline{2-42}
&25 &\cellcolor{color0}&\cellcolor{color1}&\cellcolor{color0}&\cellcolor{color0}&\cellcolor{color0}&\cellcolor{color0}&\cellcolor{color0}&\cellcolor{color0}&\cellcolor{color0}&\cellcolor{color0}&\cellcolor{color0}&\cellcolor{color0}&\cellcolor{color0}&\cellcolor{color0}&\cellcolor{color0}&\cellcolor{color0}&\cellcolor{color0}&\cellcolor{color0}&\cellcolor{color0}&\cellcolor{color0}&\cellcolor{color0}&\cellcolor{color0}&\cellcolor{color0}&\cellcolor{color0}&\cellcolor{color0}&\cellcolor{color0}&\cellcolor{color0}&\cellcolor{color0}&\cellcolor{color0}&\cellcolor{color0}&\cellcolor{color0}&\cellcolor{color0}&\cellcolor{color0}&\cellcolor{color0}&\cellcolor{color0}&\cellcolor{color0}&\cellcolor{color0}&\cellcolor{color0}&\cellcolor{color1}&\cellcolor{color0}\\
\hline
\multirow{2}{*}{\tt \$628} &26 &\cellcolor{color1}&\cellcolor{color0}&\cellcolor{color0}&\cellcolor{color0}&\cellcolor{color12}&\cellcolor{color0}&\cellcolor{color0}&\cellcolor{color0}&\cellcolor{color0}&\cellcolor{color4}&\cellcolor{color5}&\cellcolor{color5}&\cellcolor{color5}&\cellcolor{color4}&\cellcolor{color5}&\cellcolor{color5}&\cellcolor{color5}&\cellcolor{color5}&\cellcolor{color5}&\cellcolor{color4}&\cellcolor{color5}&\cellcolor{color4}&\cellcolor{color5}&\cellcolor{color4}&\cellcolor{color4}&\cellcolor{color4}&\cellcolor{color5}&\cellcolor{color4}&\cellcolor{color5}&\cellcolor{color4}&\cellcolor{color0}&\cellcolor{color0}&\cellcolor{color0}&\cellcolor{color0}&\cellcolor{color0}&\cellcolor{color12}&\cellcolor{color0}&\cellcolor{color0}&\cellcolor{color0}&\cellcolor{color1}\\
\cline{2-42}
&27 &\cellcolor{color1}&\cellcolor{color0}&\cellcolor{color0}&\cellcolor{color12}&\cellcolor{color12}&\cellcolor{color12}&\cellcolor{color12}&\cellcolor{color0}&\cellcolor{color0}&\cellcolor{color4}&\cellcolor{color5}&\cellcolor{color5}&\cellcolor{color5}&\cellcolor{color4}&\cellcolor{color5}&\cellcolor{color5}&\cellcolor{color5}&\cellcolor{color5}&\cellcolor{color5}&\cellcolor{color4}&\cellcolor{color5}&\cellcolor{color5}&\cellcolor{color5}&\cellcolor{color4}&\cellcolor{color9}&\cellcolor{color4}&\cellcolor{color5}&\cellcolor{color4}&\cellcolor{color5}&\cellcolor{color4}&\cellcolor{color0}&\cellcolor{color0}&\cellcolor{color0}&\cellcolor{color0}&\cellcolor{color12}&\cellcolor{color12}&\cellcolor{color12}&\cellcolor{color12}&\cellcolor{color0}&\cellcolor{color1}\\
\hline
\multirow{2}{*}{\tt \$650} &28 &\cellcolor{color0}&\cellcolor{color0}&\cellcolor{color0}&\cellcolor{color0}&\cellcolor{color0}&\cellcolor{color0}&\cellcolor{color0}&\cellcolor{color0}&\cellcolor{color0}&\cellcolor{color0}&\cellcolor{color0}&\cellcolor{color0}&\cellcolor{color0}&\cellcolor{color0}&\cellcolor{color0}&\cellcolor{color0}&\cellcolor{color0}&\cellcolor{color0}&\cellcolor{color0}&\cellcolor{color0}&\cellcolor{color0}&\cellcolor{color0}&\cellcolor{color0}&\cellcolor{color0}&\cellcolor{color0}&\cellcolor{color0}&\cellcolor{color0}&\cellcolor{color0}&\cellcolor{color0}&\cellcolor{color0}&\cellcolor{color0}&\cellcolor{color0}&\cellcolor{color0}&\cellcolor{color0}&\cellcolor{color0}&\cellcolor{color0}&\cellcolor{color0}&\cellcolor{color0}&\cellcolor{color0}&\cellcolor{color0}&\cellcolor{color0}&\cellcolor{color15}&\cellcolor{color0}&\cellcolor{color0}&\cellcolor{color15}&\cellcolor{color15}&\cellcolor{color0}&\cellcolor{color0}\\
\cline{2-42}
&29 &\cellcolor{color10}&\cellcolor{color10}&\cellcolor{color10}&\cellcolor{color10}&\cellcolor{color10}&\cellcolor{color10}&\cellcolor{color10}&\cellcolor{color10}&\cellcolor{color10}&\cellcolor{color10}&\cellcolor{color10}&\cellcolor{color10}&\cellcolor{color10}&\cellcolor{color10}&\cellcolor{color10}&\cellcolor{color10}&\cellcolor{color10}&\cellcolor{color10}&\cellcolor{color10}&\cellcolor{color10}&\cellcolor{color10}&\cellcolor{color10}&\cellcolor{color10}&\cellcolor{color10}&\cellcolor{color10}&\cellcolor{color10}&\cellcolor{color10}&\cellcolor{color10}&\cellcolor{color10}&\cellcolor{color10}&\cellcolor{color10}&\cellcolor{color10}&\cellcolor{color10}&\cellcolor{color10}&\cellcolor{color10}&\cellcolor{color10}&\cellcolor{color10}&\cellcolor{color10}&\cellcolor{color10}&\cellcolor{color10}&\cellcolor{color10}&\cellcolor{color15}&\cellcolor{color0}&\cellcolor{color0}&\cellcolor{color15}&\cellcolor{color15}&\cellcolor{color0}&\cellcolor{color0}\\
\hline
\multirow{2}{*}{\tt \$680} &30 &\cellcolor{color0}&\cellcolor{color1}&\cellcolor{color0}&\cellcolor{color0}&\cellcolor{color2}&\cellcolor{color2}&\cellcolor{color0}&\cellcolor{color2}&\cellcolor{color0}&\cellcolor{color2}&\cellcolor{color0}&\cellcolor{color2}&\cellcolor{color0}&\cellcolor{color2}&\cellcolor{color2}&\cellcolor{color0}&\cellcolor{color0}&\cellcolor{color3}&\cellcolor{color3}&\cellcolor{color3}&\cellcolor{color0}&\cellcolor{color3}&\cellcolor{color0}&\cellcolor{color3}&\cellcolor{color0}&\cellcolor{color3}&\cellcolor{color0}&\cellcolor{color0}&\cellcolor{color3}&\cellcolor{color0}&\cellcolor{color3}&\cellcolor{color3}&\cellcolor{color3}&\cellcolor{color0}&\cellcolor{color3}&\cellcolor{color3}&\cellcolor{color3}&\cellcolor{color0}&\cellcolor{color1}&\cellcolor{color0}\\
\cline{2-42}
&31 &\cellcolor{color0}&\cellcolor{color1}&\cellcolor{color0}&\cellcolor{color2}&\cellcolor{color2}&\cellcolor{color2}&\cellcolor{color0}&\cellcolor{color2}&\cellcolor{color0}&\cellcolor{color2}&\cellcolor{color0}&\cellcolor{color2}&\cellcolor{color0}&\cellcolor{color2}&\cellcolor{color2}&\cellcolor{color2}&\cellcolor{color0}&\cellcolor{color3}&\cellcolor{color3}&\cellcolor{color3}&\cellcolor{color0}&\cellcolor{color3}&\cellcolor{color0}&\cellcolor{color3}&\cellcolor{color0}&\cellcolor{color3}&\cellcolor{color0}&\cellcolor{color0}&\cellcolor{color3}&\cellcolor{color0}&\cellcolor{color3}&\cellcolor{color3}&\cellcolor{color3}&\cellcolor{color0}&\cellcolor{color3}&\cellcolor{color3}&\cellcolor{color3}&\cellcolor{color0}&\cellcolor{color1}&\cellcolor{color0}\\
\hline
\multirow{2}{*}{\tt \$6A8} &32 &\cellcolor{color1}&\cellcolor{color0}&\cellcolor{color13}&\cellcolor{color13}&\cellcolor{color13}&\cellcolor{color13}&\cellcolor{color0}&\cellcolor{color0}&\cellcolor{color0}&\cellcolor{color4}&\cellcolor{color4}&\cellcolor{color4}&\cellcolor{color4}&\cellcolor{color4}&\cellcolor{color4}&\cellcolor{color4}&\cellcolor{color4}&\cellcolor{color4}&\cellcolor{color4}&\cellcolor{color4}&\cellcolor{color4}&\cellcolor{color4}&\cellcolor{color4}&\cellcolor{color4}&\cellcolor{color4}&\cellcolor{color4}&\cellcolor{color4}&\cellcolor{color4}&\cellcolor{color4}&\cellcolor{color4}&\cellcolor{color4}&\cellcolor{color0}&\cellcolor{color0}&\cellcolor{color13}&\cellcolor{color13}&\cellcolor{color13}&\cellcolor{color13}&\cellcolor{color0}&\cellcolor{color0}&\cellcolor{color1}\\
\cline{2-42}
&33 &\cellcolor{color1}&\cellcolor{color0}&\cellcolor{color9}&\cellcolor{color9}&\cellcolor{color9}&\cellcolor{color9}&\cellcolor{color0}&\cellcolor{color0}&\cellcolor{color0}&\cellcolor{color4}&\cellcolor{color4}&\cellcolor{color4}&\cellcolor{color4}&\cellcolor{color4}&\cellcolor{color4}&\cellcolor{color4}&\cellcolor{color4}&\cellcolor{color4}&\cellcolor{color4}&\cellcolor{color4}&\cellcolor{color4}&\cellcolor{color4}&\cellcolor{color4}&\cellcolor{color4}&\cellcolor{color4}&\cellcolor{color4}&\cellcolor{color4}&\cellcolor{color6}&\cellcolor{color4}&\cellcolor{color6}&\cellcolor{color4}&\cellcolor{color0}&\cellcolor{color0}&\cellcolor{color9}&\cellcolor{color9}&\cellcolor{color9}&\cellcolor{color9}&\cellcolor{color0}&\cellcolor{color0}&\cellcolor{color1}\\
\hline
\multirow{2}{*}{\tt \$6D0} &34 &\cellcolor{color12}&\cellcolor{color15}&\cellcolor{color1}&\cellcolor{color4}&\cellcolor{color9}&\cellcolor{color14}&\cellcolor{color7}&\cellcolor{color0}&\cellcolor{color12}&\cellcolor{color9}&\cellcolor{color2}&\cellcolor{color1}&\cellcolor{color2}&\cellcolor{color14}&\cellcolor{color11}&\cellcolor{color2}&\cellcolor{color7}&\cellcolor{color0}&\cellcolor{color0}&\cellcolor{color0}&\cellcolor{color0}&\cellcolor{color0}&\cellcolor{color0}&\cellcolor{color0}&\cellcolor{color0}&\cellcolor{color0}&\cellcolor{color0}&\cellcolor{color0}&\cellcolor{color0}&\cellcolor{color0}&\cellcolor{color0}&\cellcolor{color0}&\cellcolor{color0}&\cellcolor{color0}&\cellcolor{color0}&\cellcolor{color0}&\cellcolor{color0}&\cellcolor{color0}&\cellcolor{color0}&\cellcolor{color0}&\cellcolor{color2}&\cellcolor{color15}&\cellcolor{color0}&\cellcolor{color0}&\cellcolor{color15}&\cellcolor{color15}&\cellcolor{color0}&\cellcolor{color0}\\
\cline{2-42}
&35 &\cellcolor{color12}&\cellcolor{color12}&\cellcolor{color12}&\cellcolor{color12}&\cellcolor{color12}&\cellcolor{color12}&\cellcolor{color12}&\cellcolor{color10}&\cellcolor{color12}&\cellcolor{color13}&\cellcolor{color13}&\cellcolor{color12}&\cellcolor{color11}&\cellcolor{color10}&\cellcolor{color12}&\cellcolor{color13}&\cellcolor{color13}&\cellcolor{color10}&\cellcolor{color10}&\cellcolor{color10}&\cellcolor{color10}&\cellcolor{color10}&\cellcolor{color10}&\cellcolor{color10}&\cellcolor{color10}&\cellcolor{color10}&\cellcolor{color10}&\cellcolor{color10}&\cellcolor{color10}&\cellcolor{color10}&\cellcolor{color10}&\cellcolor{color10}&\cellcolor{color10}&\cellcolor{color10}&\cellcolor{color10}&\cellcolor{color10}&\cellcolor{color10}&\cellcolor{color10}&\cellcolor{color10}&\cellcolor{color10}&\cellcolor{color0}&\cellcolor{color15}&\cellcolor{color0}&\cellcolor{color0}&\cellcolor{color15}&\cellcolor{color15}&\cellcolor{color0}&\cellcolor{color0}\\
\hline
\multirow{2}{*}{\tt \$700} &36 &\cellcolor{color0}&\cellcolor{color1}&\cellcolor{color0}&\cellcolor{color2}&\cellcolor{color0}&\cellcolor{color0}&\cellcolor{color0}&\cellcolor{color2}&\cellcolor{color0}&\cellcolor{color2}&\cellcolor{color0}&\cellcolor{color2}&\cellcolor{color0}&\cellcolor{color2}&\cellcolor{color0}&\cellcolor{color2}&\cellcolor{color0}&\cellcolor{color0}&\cellcolor{color3}&\cellcolor{color0}&\cellcolor{color0}&\cellcolor{color3}&\cellcolor{color0}&\cellcolor{color3}&\cellcolor{color0}&\cellcolor{color3}&\cellcolor{color3}&\cellcolor{color0}&\cellcolor{color3}&\cellcolor{color0}&\cellcolor{color3}&\cellcolor{color0}&\cellcolor{color0}&\cellcolor{color0}&\cellcolor{color3}&\cellcolor{color0}&\cellcolor{color0}&\cellcolor{color0}&\cellcolor{color1}&\cellcolor{color0}\\
\cline{2-42}
&37 &\cellcolor{color0}&\cellcolor{color1}&\cellcolor{color0}&\cellcolor{color2}&\cellcolor{color0}&\cellcolor{color0}&\cellcolor{color0}&\cellcolor{color2}&\cellcolor{color2}&\cellcolor{color2}&\cellcolor{color0}&\cellcolor{color2}&\cellcolor{color0}&\cellcolor{color2}&\cellcolor{color0}&\cellcolor{color2}&\cellcolor{color0}&\cellcolor{color0}&\cellcolor{color3}&\cellcolor{color0}&\cellcolor{color0}&\cellcolor{color3}&\cellcolor{color0}&\cellcolor{color3}&\cellcolor{color0}&\cellcolor{color3}&\cellcolor{color3}&\cellcolor{color3}&\cellcolor{color3}&\cellcolor{color0}&\cellcolor{color3}&\cellcolor{color3}&\cellcolor{color0}&\cellcolor{color0}&\cellcolor{color3}&\cellcolor{color3}&\cellcolor{color3}&\cellcolor{color0}&\cellcolor{color1}&\cellcolor{color0}\\
\hline
\multirow{2}{*}{\tt \$728} &38 &\cellcolor{color1}&\cellcolor{color0}&\cellcolor{color1}&\cellcolor{color1}&\cellcolor{color1}&\cellcolor{color1}&\cellcolor{color1}&\cellcolor{color0}&\cellcolor{color0}&\cellcolor{color4}&\cellcolor{color5}&\cellcolor{color5}&\cellcolor{color5}&\cellcolor{color4}&\cellcolor{color5}&\cellcolor{color5}&\cellcolor{color5}&\cellcolor{color5}&\cellcolor{color5}&\cellcolor{color4}&\cellcolor{color5}&\cellcolor{color5}&\cellcolor{color5}&\cellcolor{color4}&\cellcolor{color4}&\cellcolor{color4}&\cellcolor{color4}&\cellcolor{color6}&\cellcolor{color4}&\cellcolor{color6}&\cellcolor{color4}&\cellcolor{color0}&\cellcolor{color0}&\cellcolor{color1}&\cellcolor{color1}&\cellcolor{color1}&\cellcolor{color1}&\cellcolor{color1}&\cellcolor{color0}&\cellcolor{color1}\\
\cline{2-42}
&39 &\cellcolor{color1}&\cellcolor{color0}&\cellcolor{color2}&\cellcolor{color2}&\cellcolor{color2}&\cellcolor{color2}&\cellcolor{color2}&\cellcolor{color0}&\cellcolor{color0}&\cellcolor{color4}&\cellcolor{color5}&\cellcolor{color5}&\cellcolor{color5}&\cellcolor{color4}&\cellcolor{color5}&\cellcolor{color5}&\cellcolor{color5}&\cellcolor{color5}&\cellcolor{color5}&\cellcolor{color4}&\cellcolor{color5}&\cellcolor{color4}&\cellcolor{color5}&\cellcolor{color4}&\cellcolor{color5}&\cellcolor{color5}&\cellcolor{color4}&\cellcolor{color6}&\cellcolor{color4}&\cellcolor{color6}&\cellcolor{color4}&\cellcolor{color0}&\cellcolor{color0}&\cellcolor{color2}&\cellcolor{color2}&\cellcolor{color2}&\cellcolor{color2}&\cellcolor{color2}&\cellcolor{color0}&\cellcolor{color1}\\
\hline
\multirow{2}{*}{\tt \$750} &40 &\cellcolor{color0}&\cellcolor{color0}&\cellcolor{color0}&\cellcolor{color0}&\cellcolor{color0}&\cellcolor{color0}&\cellcolor{color0}&\cellcolor{color0}&\cellcolor{color0}&\cellcolor{color0}&\cellcolor{color0}&\cellcolor{color0}&\cellcolor{color0}&\cellcolor{color0}&\cellcolor{color0}&\cellcolor{color0}&\cellcolor{color0}&\cellcolor{color0}&\cellcolor{color0}&\cellcolor{color0}&\cellcolor{color0}&\cellcolor{color0}&\cellcolor{color0}&\cellcolor{color0}&\cellcolor{color0}&\cellcolor{color0}&\cellcolor{color0}&\cellcolor{color0}&\cellcolor{color0}&\cellcolor{color0}&\cellcolor{color0}&\cellcolor{color0}&\cellcolor{color0}&\cellcolor{color0}&\cellcolor{color0}&\cellcolor{color0}&\cellcolor{color0}&\cellcolor{color0}&\cellcolor{color0}&\cellcolor{color0}&\cellcolor{color0}&\cellcolor{color15}&\cellcolor{color0}&\cellcolor{color0}&\cellcolor{color15}&\cellcolor{color15}&\cellcolor{color0}&\cellcolor{color0}\\
\cline{2-42}
&41 &\cellcolor{color10}&\cellcolor{color10}&\cellcolor{color10}&\cellcolor{color10}&\cellcolor{color10}&\cellcolor{color10}&\cellcolor{color10}&\cellcolor{color10}&\cellcolor{color10}&\cellcolor{color10}&\cellcolor{color10}&\cellcolor{color10}&\cellcolor{color10}&\cellcolor{color10}&\cellcolor{color10}&\cellcolor{color10}&\cellcolor{color10}&\cellcolor{color10}&\cellcolor{color10}&\cellcolor{color10}&\cellcolor{color10}&\cellcolor{color10}&\cellcolor{color10}&\cellcolor{color10}&\cellcolor{color10}&\cellcolor{color10}&\cellcolor{color10}&\cellcolor{color10}&\cellcolor{color10}&\cellcolor{color10}&\cellcolor{color10}&\cellcolor{color10}&\cellcolor{color10}&\cellcolor{color10}&\cellcolor{color10}&\cellcolor{color10}&\cellcolor{color10}&\cellcolor{color10}&\cellcolor{color10}&\cellcolor{color10}&\cellcolor{color10}&\cellcolor{color15}&\cellcolor{color0}&\cellcolor{color0}&\cellcolor{color15}&\cellcolor{color15}&\cellcolor{color0}&\cellcolor{color0}\\
\hline
\multirow{2}{*}{\tt \$780} &42 &\cellcolor{color0}&\cellcolor{color1}&\cellcolor{color0}&\cellcolor{color2}&\cellcolor{color0}&\cellcolor{color0}&\cellcolor{color0}&\cellcolor{color2}&\cellcolor{color2}&\cellcolor{color2}&\cellcolor{color0}&\cellcolor{color2}&\cellcolor{color0}&\cellcolor{color2}&\cellcolor{color2}&\cellcolor{color2}&\cellcolor{color0}&\cellcolor{color0}&\cellcolor{color3}&\cellcolor{color0}&\cellcolor{color0}&\cellcolor{color3}&\cellcolor{color0}&\cellcolor{color3}&\cellcolor{color0}&\cellcolor{color3}&\cellcolor{color0}&\cellcolor{color3}&\cellcolor{color3}&\cellcolor{color0}&\cellcolor{color3}&\cellcolor{color3}&\cellcolor{color0}&\cellcolor{color0}&\cellcolor{color3}&\cellcolor{color3}&\cellcolor{color3}&\cellcolor{color0}&\cellcolor{color1}&\cellcolor{color0}\\
\cline{2-42}
&43 &\cellcolor{color0}&\cellcolor{color1}&\cellcolor{color0}&\cellcolor{color2}&\cellcolor{color0}&\cellcolor{color0}&\cellcolor{color0}&\cellcolor{color2}&\cellcolor{color0}&\cellcolor{color2}&\cellcolor{color0}&\cellcolor{color2}&\cellcolor{color0}&\cellcolor{color2}&\cellcolor{color2}&\cellcolor{color0}&\cellcolor{color0}&\cellcolor{color0}&\cellcolor{color3}&\cellcolor{color0}&\cellcolor{color0}&\cellcolor{color3}&\cellcolor{color0}&\cellcolor{color3}&\cellcolor{color0}&\cellcolor{color3}&\cellcolor{color0}&\cellcolor{color3}&\cellcolor{color3}&\cellcolor{color0}&\cellcolor{color3}&\cellcolor{color0}&\cellcolor{color0}&\cellcolor{color0}&\cellcolor{color0}&\cellcolor{color0}&\cellcolor{color3}&\cellcolor{color0}&\cellcolor{color1}&\cellcolor{color0}\\
\hline
\multirow{2}{*}{\tt \$7A8} &44 &\cellcolor{color1}&\cellcolor{color0}&\cellcolor{color0}&\cellcolor{color6}&\cellcolor{color6}&\cellcolor{color6}&\cellcolor{color0}&\cellcolor{color0}&\cellcolor{color0}&\cellcolor{color4}&\cellcolor{color5}&\cellcolor{color5}&\cellcolor{color5}&\cellcolor{color4}&\cellcolor{color5}&\cellcolor{color5}&\cellcolor{color5}&\cellcolor{color5}&\cellcolor{color5}&\cellcolor{color4}&\cellcolor{color5}&\cellcolor{color4}&\cellcolor{color5}&\cellcolor{color4}&\cellcolor{color5}&\cellcolor{color5}&\cellcolor{color4}&\cellcolor{color6}&\cellcolor{color4}&\cellcolor{color6}&\cellcolor{color4}&\cellcolor{color0}&\cellcolor{color0}&\cellcolor{color0}&\cellcolor{color6}&\cellcolor{color6}&\cellcolor{color6}&\cellcolor{color0}&\cellcolor{color0}&\cellcolor{color1}\\
\cline{2-42}
&45 &\cellcolor{color1}&\cellcolor{color0}&\cellcolor{color0}&\cellcolor{color0}&\cellcolor{color0}&\cellcolor{color0}&\cellcolor{color0}&\cellcolor{color0}&\cellcolor{color0}&\cellcolor{color4}&\cellcolor{color5}&\cellcolor{color5}&\cellcolor{color5}&\cellcolor{color4}&\cellcolor{color5}&\cellcolor{color5}&\cellcolor{color5}&\cellcolor{color5}&\cellcolor{color5}&\cellcolor{color4}&\cellcolor{color5}&\cellcolor{color5}&\cellcolor{color5}&\cellcolor{color4}&\cellcolor{color4}&\cellcolor{color4}&\cellcolor{color4}&\cellcolor{color4}&\cellcolor{color4}&\cellcolor{color4}&\cellcolor{color4}&\cellcolor{color0}&\cellcolor{color0}&\cellcolor{color0}&\cellcolor{color0}&\cellcolor{color0}&\cellcolor{color0}&\cellcolor{color0}&\cellcolor{color0}&\cellcolor{color1}\\
\hline
\multirow{2}{*}{\tt \$7D0} &46 &\cellcolor{color0}&\cellcolor{color0}&\cellcolor{color0}&\cellcolor{color0}&\cellcolor{color0}&\cellcolor{color0}&\cellcolor{color0}&\cellcolor{color0}&\cellcolor{color0}&\cellcolor{color0}&\cellcolor{color0}&\cellcolor{color0}&\cellcolor{color0}&\cellcolor{color0}&\cellcolor{color0}&\cellcolor{color0}&\cellcolor{color0}&\cellcolor{color0}&\cellcolor{color0}&\cellcolor{color0}&\cellcolor{color0}&\cellcolor{color0}&\cellcolor{color0}&\cellcolor{color0}&\cellcolor{color0}&\cellcolor{color0}&\cellcolor{color0}&\cellcolor{color0}&\cellcolor{color0}&\cellcolor{color0}&\cellcolor{color0}&\cellcolor{color0}&\cellcolor{color0}&\cellcolor{color0}&\cellcolor{color0}&\cellcolor{color0}&\cellcolor{color0}&\cellcolor{color0}&\cellcolor{color0}&\cellcolor{color0}&\cellcolor{color0}&\cellcolor{color15}&\cellcolor{color0}&\cellcolor{color0}&\cellcolor{color15}&\cellcolor{color15}&\cellcolor{color0}&\cellcolor{color0}\\
\cline{2-42}
&47 &\cellcolor{color10}&\cellcolor{color10}&\cellcolor{color10}&\cellcolor{color10}&\cellcolor{color10}&\cellcolor{color10}&\cellcolor{color10}&\cellcolor{color10}&\cellcolor{color10}&\cellcolor{color10}&\cellcolor{color10}&\cellcolor{color10}&\cellcolor{color10}&\cellcolor{color10}&\cellcolor{color10}&\cellcolor{color10}&\cellcolor{color10}&\cellcolor{color10}&\cellcolor{color10}&\cellcolor{color10}&\cellcolor{color10}&\cellcolor{color10}&\cellcolor{color10}&\cellcolor{color10}&\cellcolor{color10}&\cellcolor{color10}&\cellcolor{color10}&\cellcolor{color10}&\cellcolor{color10}&\cellcolor{color10}&\cellcolor{color10}&\cellcolor{color10}&\cellcolor{color10}&\cellcolor{color10}&\cellcolor{color10}&\cellcolor{color10}&\cellcolor{color10}&\cellcolor{color10}&\cellcolor{color10}&\cellcolor{color10}&\cellcolor{color10}&\cellcolor{color15}&\cellcolor{color0}&\cellcolor{color0}&\cellcolor{color15}&\cellcolor{color15}&\cellcolor{color0}&\cellcolor{color0}\\
\hline
\end{tabular}
\end{table*}
\end{tiny}


%\end{document}
