

If you are making a game or a demo for the old Apple II, you will want
some sound.

Having a built-in speaker was a nice feature in 1977 when the Apple II
first launched, but your output audio is a bit limited as it is driven
by a flip-flop that you need to bitbang (and there are no timers or
interrupt sources to help like there was on the IBM PC).
(Note, we are discussing the first few generations of Apple II products here,
the IIgs which came out in 1987(?) had a built-in advanced sound chip
but that is another story for another day).

Despite that you can do some interesting sound, including sampled audio
(that would take up a  good fraction of a flopp disk, like in the game
X) or even two-channel ``Electric Duet'' (which used up most of the CPU
power leaving not much time for anything else).

The Apple II did have 7 expansion slots, and so many sound boards were
made for it.  None became a standard though, so only a small fraction
of games released for the platform.
One of the more common was the Mockingboard.

A Mockingboard card has two AY-3-8910 sound chips on it, and optionally
had speech chips (type).  You will often find empty sockets there as they
are xpensive these days.

The AY-3-8910 was common at the time, made by X who went on to become
Microchip (of PIC fame).
The chip was found in many arcade cabinets, as well as 
the Atari ST and some models of ZX Spectrum.
It can make three channels of square waves, with noise applied, and
one hadrware evenlope.
It can make for some nice sound in skilled hands, although the Commodore
readers are probably already scoffing in comparison to their SID chips
(and we won't even start with the Amiga fans).

Programming the AY-3-8910 is fairly straightforward, the diagram is in Figure
X.  There are 16 registers, but the top two are only used for GPIO lines
on the earlier 40-pin versions of the chip.

You have registers for the audio period of channel A, B, and C, 12-bits.
Then a 4-bit amplitude setting for each channel (logarithmic).
Then a noise channel, a mixer to pick what noise/audio channels are playing.
Then the preiod and type for the hardware envelope.

Writing the registers is straightforward, using three wires with a protocol
easy enough to do in small hacking projects.  Typically with some shift
registers and SPI. 
The Mockingboard has two AY-3-8910 chips (allowing 6-channels, split
left/right stereo).  It uses two 6522 I/O chips to interface with the
Apple II bus, and they also provide a timer interrupt source (something
a stock Apple II lacks).

One complication is frequency.  The Apple II drives it at 1MHz, but for
example the ZX runs it at 1.77MHz, so the frequencies will be different
unless you translate before hand.

TODO: diagram of chip


Related Work/Earlier

	French-touch

	guy playing on Apple II with z80

	YM5 files, limitation

Vortex Tracker format

	By a Russian, russian docs, Delphi/Pascal.


Research
	First to C on Raspberry Pi
	Validation against AY_emul
	Then to 6502 assembly
	Decode to screen using COUT, re-direct to "printer" on emulator,
	take text output and diff against proper output with Linux tool.

Final Output

